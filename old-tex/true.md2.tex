\hypertarget{true-truer-truest}{%
\section{True, Truer, Truest}\label{true-truer-truest}}

What the world needs now is another theory of vagueness. Not because the
old theories are useless. Quite the contrary, the old theories provide
many of the materials we need to construct the truest theory of
vagueness ever seen. The theory shall be similar in motivation to
supervaluationism, but more akin to many-valued theories in
conceptualisation. What I take from the many-valued theories is the idea
that some sentences can be \emph{truer} than others. But I say very
different things to the ordering over sentences this relation generates.
I say it is not a linear ordering, so it cannot be represented by the
real numbers. I also argue that since there is higher-order vagueness,
any mapping between sentences and mathematical objects is bound to be
inappropriate. This is no cause for regret; we can say all we want to
say by using the comparative \emph{truer than} without mapping it onto
some mathematical objects. From supervaluationism I take the idea that
we can keep classical logic without keeping the familiar bivalent
semantics for classical logic. But my preservation of classical logic is
more comprehensive than is normally permitted by supervaluationism, for
I preserve classical inference rules as well as classical sequents. And
I do this without relying on the concept of acceptable precisifications
as an unexplained explainer.

The world does not need another guide to varieties of theories of
vagueness, especially since Timothy \citet{Williamson1994-WILV} and
Rosanna \citet{Keefe2000} have already provided quite good guides. I
assume throughout familiarity with popular theories of vagueness.

\hypertarget{truer}{%
\subsection{Truer}\label{truer}}

The core of my theory is that some sentences involving vague terms are
truer than others. I won't give an analysis of \emph{truer}, instead I
will argue that we already tacitly understand this relation. The main
argument for this will turn on a consideration of two `many-valued'
theories of vagueness, one of which will play a central role (as the
primary villain) in what follows.

The most familiar many-valued theory, call it \(M\), says there are
continuum many truth values, and they can be felicitously represented by
the interval \([0,~1]\). The four main logical connectives: \emph{and},
\emph{or}, \emph{if} and \emph{not} are truth-functional. The functions
are:

\[\begin{aligned}
V(A \wedge B) &= min(V(A), V(B))  \hspace*{\fill} \\
V(A \vee B) &= max(V(A), V(B))  \hspace*{\fill} \\
V(A \rightarrow B) &= max(1, 1 - V(A) + V(B))  \hspace*{\fill} \\
V(\neg A) &= 1 - V(A)\end{aligned}\]

where \(V\) is the valuation function on sentences, \(min(x, y)\) is the
smaller of \(x\) and \emph{y} and \(max(x, y)\) is the larger of \(x\)
and \emph{y}.

Adopting these rules for the connectives commits us to adopting the
logic {Ł}\textsubscript{C}. \(M\) is the theory that this semantic
model, under its most natural interpretation, is appropriate for vague
natural languages. (We'll discuss less natural interpretations
presently.)

\(M\) tells a particularly nice story about the Sorites. A premise like
\emph{If she's rich, someone with just a little less money is also rich}
will have a very high truth value. If we make the difference in money
between the two subjects small enough, this conditional will have a
truth value arbitrarily close to 1.

\(M\) also tells a nice story about borderline cases and
determinateness. An object \(a\) is a borderline case of being an \(F\)
just in case the sentence \emph{a is F} has a truth value between 0 and
1 exclusive. Similarly, \(a\) is a determinate \(F\) just in case the
truth value of \emph{a is F} is 1. (It is worthwhile comparing how
simple this analysis of determinateness is to the difficulties
supervaluationists have in providing an analysis of determinateness. On
this topic, see \citet{Williamson1995}, \citet{McGee1998} and
\citet{Williamson2004-WILRTM}.)

But \(M\) tells a particularly implausible story about contradictions.
Here is how Timothy \citet{Williamson1994-WILV} makes this problem
vivid.

\begin{quote}
More disturbing is that the law of non-contradiction fails {\ldots{}}.
\(\neg(p \wedge \neg p)\) always has the same degree of truth as
\(p \vee \neg p\), and thus is perfectly true only when \(p\) is either
perfectly true or perfectly false. When \(p\) is half-true, so are both
\(p \wedge \neg p\) and \(\neg(p \wedge \neg p)\).
\cite[118]{Williamson1994-WILV}

At some point {[}in waking up{]} `He is awake' is supposed to be
half-true, so `He is not awake' will be half-true too. Then `He is awake
and he is not awake' will count as half-true. How can an explicit
contradiction be true to any degree other than 0?
\cite[136]{Williamson1994-WILV}
\end{quote}

There is a way to keep the semantic engine behind \(M\) while avoiding
this consequence. (The following few paragraphs are indebted pretty
heavily to the criticisms of Strawson's theory of descriptions in
\citet{Dummett1959})

Consider an interpretation of the above semantics on which there are
only two truth values: True and False. Any sentence that gets truth
value 1 is true, all the others are false. The numbers in {[}0, 1)
represent different ways of being false. (As Tolstoy might have put it,
all true sentences are alike, but every false sentence is false in its
own unique way.) Which way a sentence is false can affect the truth
value of compounds containing that sentence. In particular, if \(A\) and
\(B\) are false, then the truth values of \emph{Not A} and \emph{If A
then B} will depend on the ways \(A\) and \(B\) take their truth values.
If \(V\)(\(A\)) = 0 and \(V\)(\(B\)) = 0.3, then \emph{Not A} and
\emph{If A then B} will be true, but if \(V\)(\(A\)) becomes 0.6, and
remember this is just another way of being false, both \emph{Not A} and
\emph{If A then B} will be false.

The new theory we get, one I'll call \(M_D\), is similar to \(M\) in
some respects. For example, it agrees about what the axioms should be
for a logic for natural language. But it has several philosophical
differences. In particular, it has none of the three characteristics of
\(M\) we noted above.

It cannot tell as plausible story as \(M\) does about the Sorites. If
any sentence with truth value below 1 is false, then many of the
premises in a Sorites argument are \emph{false}. This is terrible -- it
was bad enough to be told that one of the premises were false, but now
we find many thousands of them are false. I doubt that being told they
are false in a distinctive way will improve our estimation of the
theory. Similarly, it is hard to see just how the new theory has
anything interesting to say about the concept of a borderline case.

On the other hand, according to \(M_D\), contradictions are always
false. To be sure, a contradiction might be false in some obscure new
way, but it is still false. Recall Williamson's objection that an
explicit contradiction should be true to degree 0 and nothing more. This
objection only works if being true to degree 0.5 is meant to be
semantically significant. If being `true to degree 0.5' is just another
way of being false, then there is presumably nothing wrong with
contradictions are true to degree 0.5. This is not to say Williamson's
objection is no good, since he intended it as an objection to \(M\), but
just to say that re-interpreting the semantic significance of the
numbers in \(M\) makes a philosophical difference.

Despite \(M_D\)'s preferable treatment of contradictions, I think \(M\)
is overall a better theory because it has a much better account of
borderline cases. But for now I want to stress a simpler point: \(M\)
and \(M_D\) are \emph{different} theories of vagueness, and that we
grasp the difference between these theories. One crucial difference
between the two theories is that in \(M\), but not \(M_D\), \(S_1\)is
truer than \(S_2\) if \(V\)(\(S_1\)) is greater than \(V\)(\(S_2\)). In
\(M_D\), if \(S_1\)is truer than \(S_2\), \(V\)(\(S_1\)) must be one and
\(V\)(\(S_2\)) less than one. And that is the \emph{only} difference
between the two theories. So if we understand this difference, we must
grasp this concept \emph{truer than}. Indeed, it is in virtue of
grasping this concept that we understand why saying each of the Sorites
conditionals is almost true is a \emph{prima facie} plausible response
to the Sorites, and why having a theory that implies contradictions are
truer than many other sentences is a rather embarrassing thing.

I have implicitly defined \emph{truer} by noting its theoretical role.
As David \citet{Lewis1972a} showed, terms can be implicitly defined by
their theoretical role. There is one unfortunate twist here in that
\emph{truer} is defined by its role in a \emph{false} theory, but that
does not block the implicit definition story. We know what
\emph{phlogiston} and \emph{ether} mean because of their role in some
false theories. The meaning of \emph{truer} can be extracted in the same
way from the false theory \(M\).

\hypertarget{further-reflections-on-truer}{%
\subsection{\texorpdfstring{Further Reflections on
\emph{Truer}}{Further Reflections on Truer}}\label{further-reflections-on-truer}}

As noted, I won't give a reductive analysis of \emph{truer}. The hopes
for doing that are no better than the hopes of giving a reductive
analysis of \emph{true}. But I will show that we pre-theoretically
understand the concept.

My primary argument for this has already been given. Intuitively we do
understand the difference between \(M\) and its \(M_D\), and this is
only explicable by our understanding \emph{truer}. Hence we understand
\emph{truer}.

Second, it's noteworthy that \emph{truer} is morphologically complex. If
we understand \emph{true}, and understand the modifier -\emph{er}, then
we know enough in principle to know how they combine. Not every
predicate can be turned into a comparative. But most can, and our
default assumption should be that \emph{true} is like the majority.

I have heard two arguments against that assumption. First, it could be
argued that most comparatives in English generate linear orderings, but
\emph{truer} generates a non-linear ordering. I reject the premise of
this argument. \emph{Cuter}, \emph{Smarter}, \emph{Smellier}, and
\emph{Tougher} all generate non-linear orderings over their respective
domains, and they seem fairly indicative of large classes. Second, it
could be argued that it's crucial to understanding comparatives that we
understand the interaction of the underlying adjectives with comparison
classes. Robin Jeshion and Mike Nelson made this objection in their
comments on my paper at BSPC 2003. Again, the premise is not obviously
true. We can talk about some objects being \emph{straighter} or
\emph{rounder} despite the fact that it's hard to understand \emph{round
for an office building} or \emph{straight for a line drive}. (Jonathan
Bennett made this point in discussion at BSPC.) \emph{Straight} and
\emph{round} either don't have or don't need comparison classes, but
they form comparatives. So \emph{true}, which also does not take
comparison classes, could also form a comparative.

Finally, if understanding the inferential role of a logical operator
helps know its meaning, then it is notable that \emph{truer} has a very
clear inferential role. It is the same as a strict material implication
\(\square (q \supset p)\) defined using a necessity operator whose logic
is KT. Since many operators have just this logic, this doesn't
individuate \emph{truer}, but it helps with inferential role semantics
aficionados.

I claim that the concept \emph{truer}, and the associated concept
\emph{as true as}, are the only theoretical tools we need to provide a
complete theory of vagueness. It is simplest to state the important
features of my theory by contrasting it with \(M\). I keep the following
good features of \(M\).

\begin{description}
\item[G1]
There are \emph{intermediate} sentences, i.e. sentences that are truer
than some sentences and less true than others. For definiteness, I will
say \(S\) is intermediate iff \(S\) is truer than 0=1 and less true than
0=0.
\item[G2]
\(a\) is a borderline \(F\) iff \emph{a is F} is intermediate, and \(a\)
is determinately \(F\) iff \(a\) is \(F\) and \(a\) is not a borderline
\(F\).
\end{description}

I won't repeat the arguments here, but I take G1 to be a large advantage
of theories like \(M\) over epistemicist theories. (See
\citet{Burgess2001}, \citet{Sider2001} and \citet{Weatherson2003-WEAEPA}
for more detailed arguments to this effect.) And as noted G2 is a much
simpler analysis of determinacy and borderline than supervaluationists
have been able to offer.

I drop the following bad features of \(M\).

\begin{description}
\item[B1 \emph{Some contradictions are intermediate sentences}.]
\hfill\break
On my theory all contradictions are determinately false, and
determinately determinately false, and so on. The argument for this has
been given above.
\item[B2 \emph{Some classical tautologies are intermediate sentences}.]
\hfill\break
On my theory all classical tautologies are determinately true, and
determinately determinately true, and so on. We will note three
arguments for this being an improvement in the next section.
\item[B3 \emph{Some classical inference rules are inadmissible}.]
\hfill\break
On my theory all classical inference rules are admissible. As
\citet{Williamson1994-WILV} showed, the most prominent version of
supervaluationism is like \(M\) in ruling some classical rules to be
inadmissible, and this is clearly a cost of those theories.
\item[B4 \emph{Sentences of the form} S is intermediate \emph{are never
intermediate}]
\hfill\break
I will argue below this is a consequence of \(M\), and it means it is
impossible to provide a plausible theory of higher-order vagueness
within \(M\). In my theory we can say that there is higher-order
vagueness by treating \emph{truer} as an iterable operator, so we can
say that \emph{S is intermediate} is intermediate. If \(S\) is \emph{a
is F}, that's equivalent to saying that \(a\) is a borderline case of a
borderline case of an \(F\). Essentially we get out theory of
higher-order vagueness by simply iterating our theory of first-order
vagueness, which is what Williamson does in his justly celebrated
treatment of higher-order vagueness. Note it's not just \(M\) that has
troubles with higher-order vagueness. See \citet{Williamson1994-WILV}
and \citet{Weatherson2003-Keefe} for the difficulties supervaluationists
have with higher-order vagueness. The treatment of higher-order
vagueness here is a substantial advantage of my theory over
supervaluationism.
\item[B5 \emph{Truer is a linear relation}.]
\hfill\break
On my theory it need not be the case that \(S_1\)is truer than \(S_2\),
or \(S_2\) is truer than \(S_1\), or they are as true as each other. In
the last section I will argue that this is a substantial advantage of my
theory. I claim that \emph{truer} generates a Boolean lattice on
possible sentences of the language. (For a familiar example of a Boolean
lattice, think of the subsets of \(\mathbb{R}\) ordered by the subset
relation.)
\end{description}

I also provide a very different, and much more general, treatment of the
Sorites than is available within \(M\). The biggest technical difference
between my theory and \(M\) concerns the relationship between the
semantics and the logic. In \(M\) the logic falls out from the
truth-tables. Since I do not have the concept of an intermediate truth
value in my theory, I could not provide anything like a truth-table.
Instead I posit several constraints on the interaction of \emph{truer}
with familiar connectives, posit an analysis of validity in terms of
\emph{truer}, and note that those two posits imply that all and only
classically admissible inference rules are admissible.

\hypertarget{constraints-on-truer-and-classical-logic}{%
\subsection{Constraints on Truer and Classical
Logic}\label{constraints-on-truer-and-classical-logic}}

The following ten constraints on \emph{truer} seem intuitively
compelling. I've listed here both the philosophically important informal
claim, and the formal interpretation of that claim. (I use
\(A \geqslant _T B\) as shorthand for \(A\) \emph{is at least as}
\emph{true as B}. Note all the quantifiers over sentences here are
\emph{possibilist} quantifiers, we quantify over all possible sentences
in the language.)

\begin{description}
\item[(A1)]
\(\geqslant _T\) is a weak ordering (i.e. reflexive and transitive)\\
If \(A \geqslant _T B\) and \(B \geqslant _T C\) then
\(A \geqslant _T C\)\\
\(A \geqslant _T A\)
\item[(A2)]
\(\wedge\) is a greatest lower bound with respect to \(\geqslant _T\)\\
\(A \wedge B \geqslant _T C\) iff \(A \geqslant _T C\) and
\(B \geqslant _T C\)\\
\(C \geqslant _T\) \emph{A}\(\wedge B\) iff for all \(S\) such that
\(A \geqslant _T S\) and \(B \geqslant _T\)\emph{S} it is also the case
that \(C \geqslant _T S\)
\item[(A3)]
\(\vee\) is a least upper bound with respect to \(\geqslant _T\)\\
\(A \vee B \geqslant _T C\) iff for all \(S\) such that
\(S \geqslant _T A\)~and \(S \geqslant _T B\), it is also the case that
\(S \geqslant _T C\)\\
\(C \geqslant _T\) \emph{A}\(\vee B\) iff \(C \geqslant _T A\) and
\(B \geqslant _T C\)
\item[(A4)]
\(\neg\) is ordering inverting with respect to \(\geqslant _T\)\\
\(A \geqslant _T B\) iff \(\neg B \geqslant _T \neg A\)
\item[(A5)]
Double negation is redundant\\
\(\neg \neg A =_T A\)
\item[(A6)]
There is an absolutely false sentence \(S_F\) and an absolutely true
sentence \(S_T\)\\
There are sentences \(S_F\)and \(S_T\)such that \(S_F =_T \neg S_T\) and
\(\neg S_F =_T S_T\)~ and for all \(S\):
\(S_T \geqslant _T S \geqslant _T S_F\)
\item[(A7)]
Contradictions are absolutely false\\
\(A \wedge \neg A =_T S_F\)
\item[(A8)]
\(\forall\) is a greatest lower bound with respect to \(\geqslant _T\)\\
\(A \geqslant _T \forall x\)(\(\phi x\))~iff for all \(S\) such that for
all \(o\), if \(n\) is a name of \(o\) then \(\phi n \geqslant _T S\),
it is the case that \(A \geqslant _T S\)\\
\(\forall x\)(\(\phi x\))~\(\geqslant _T A\) iff for all \(o\), if \(n\)
is a name of \(o\) then \(\phi n \geqslant _T A\)
\item[(A9)]
\(\exists\) is a least upper bound with respect to \(\geqslant _T\)\\
\(A \geqslant _T \exists x\)(\(\phi x\))~iff for all \(o\), if \(n\) is
a name of \(o\) then
\(A \geqslant _T \phi\)\emph{n}\(\exists x\)(\(\phi x\))~\(\geqslant _T A\)
iff for all \(S\) such that for all \(o\), if \(n\) is a name of \(o\)
then \(S \geqslant _T \phi n\), \(S \geqslant _T A\)
\item[(A10)]
A material implication with respect to \(\geqslant _T\) can be
defined.\\
There is an operative \(\rightarrow\) such that

\begin{enumerate}
\item
  \(B \rightarrow A \geqslant _T S_T\)iff \(A \geqslant _T B\)
\item
  (\(A \wedge B\))\(\rightarrow C =_T A  \rightarrow\)(\(B \rightarrow C\))
\end{enumerate}
\end{description}

Apart from (A10) these are fairly straightforward. We can't argue for
(A10) by saying English \emph{if{\ldots{}}then} is a material
implication, because that leads directly to the paradoxes of material
implication. Assuming that \(\neg A \vee B\) is a material implication
is equivalent to assuming (inter alia) that \(A \vee \neg A\) is
perfectly true. I believe this, but since it is denied by many I want
that to be a conclusion, not a premise. So the argument for (A10) must
be a little indirect. In particular, we will appeal to the behaviour of
quantifiers. We can formally represent \emph{All Fs are Gs} in two ways:
using restricted or unrestricted quantifiers. In the first case the
formal representation will look like:

\begin{quote}
\(\forall x\)(\emph{Fx} ? \emph{Gx})
\end{quote}

with some connective in place of `?'. But it seems clear that whatever
connective goes in there must be a material implication. In the second
case, the formal representation will look like:

\begin{quote}
{[}\(\forall x\): \emph{Fx}{]} \emph{Gx}
\end{quote}

In that case, we can define a connective \(\nabla\) that satisfies the
definition of a material implication:

\begin{quote}
\(A \nabla B\)~=\textsubscript{df} {[}\(\forall x\):
\(A \wedge x\)=\(x\){]} (\(B \wedge x\)=\(x\))
\end{quote}

This is equivalent to the odd (but intelligible) sentence
\emph{Everything such that A is such that B}. Again, considerations
about what should be logical truths involving quantifiers suggests that
\(\nabla\) must be a material implication. So either way there should be
a material implication present in the language, as (A10) says.

Given (A1) to (A10) it follows that this material implication is
equivalent to \(\neg A \vee B\), and hence \(A \vee \neg A\) is a
logical truth. This is a surprising \emph{conclusion}, since intuitively
vagueness poses problems for excluded middle, but I think it is more
plausible that vague instances of excluded middle are problematic for
\emph{pragmatic} reasons than that any of (A1) to (A10) are false.

What is interesting about these ten constraints is that they suffice for
classical logic, with just one more supposition. I assume that an
argument is \emph{valid} iff it is impossible for the premises taken
collectively to be truer than the conclusion, i.e. iff it is impossible
for the conjunction of the premises to be truer than the conclusion.
Given that, we get:

\begin{quote}
\(\forall A_1\), {\ldots{}},\(A_n\),\(B\): \(A_1\), {\ldots{}},
\(A_n \vdash\)\textsubscript{T}~\(B\) iff, according to classical logic,
\(A_1\), {\ldots{}},~\(A_n\) \(\vdash\) \(B\)
\end{quote}

(I use \(\Gamma\) \(\vdash\)\textsubscript{T} \(A\) to mean that in all
models for \(\geqslant _T\) that satisfy the constraints, here (A1) to
(A10), the conclusion is at least as true as the greatest lower bound of
the premises.) I won't prove this result, but the idea is that (A1) to
(A10) imply that \(\geqslant _T\) defines a Boolean lattice over
equivalence classes of sentences with respect to \(=_T\). And all
Boolean lattices are models for classical logic, from which our result
follows. Indeed, Boolean lattices are models for classical logic in the
strong sense that classical inference rules, such as conditional proof
and \emph{reductio}, are admissible in logics defined on them, so we
also get the admissibility of classical inference rules in this theory.
(Note that this result only holds in the right-to-left direction for
languages that contain the \(\geqslant _T\) operator. Once this operator
is added, some arguments that are not classically valid, such as
\(B \geqslant _T A\), \(A \vdash\)\textsubscript{T}~\(B\), will valid.
But the addition of this operator is conservative: if we look at the
\(\geqslant _T\)-free fragment of such languages, the above result still
holds in both directions.)

There are three reasons for wanting to keep classical logic in a theory
of vagueness. First, as Williamson has stressed, classical logic is at
the heart of many successful research programs. Second, non-classical
theories of vagueness tend to abandon too much of classical logic. For
instance, \(M\) abandons the very plausible schema
(\(A \wedge A \rightarrow B\))\(\rightarrow B\). The third reason is the
one given here - these ten independently plausible constraints on
\emph{truer} entail that the logic for a language containing
\emph{truer} should be classical. These three arguments add up to a
powerful case that non-classical theories like \(M\) are mistaken, and
we should prefer a theory that preserved classical logic.

\hypertarget{semantics-and-proof-theory}{%
\subsection{Semantics and Proof
Theory}\label{semantics-and-proof-theory}}

In this section I will describe a semantics and proof theory for a
language containing \emph{truer} as an iterable operator. This is
important for the theory of higher-order vagueness. I say that \(a\) is
a borderline borderline \(F\) just in case the sentence \emph{a is a
borderline F} is intermediate, where `borderline' is analysed as in
section 2. It might not be obvious that it is consistent with (A1) to
(A10) that any sentence \emph{a is a borderline F} could be consistent.
One virtue of the model theory outlined here is that it shows this is
consistent.

For comparison, note that \(M\) as it stands has no way of dealing with
higher-order vagueness, i.e. with borderline cases of borderline cases
of \(F\)-ness. If every sentence \emph{a is a borderline F} either does
or does not receive an integer truth value, then this intuitive
possibility is ruled out. We cannot solve the problem simply by
iterating \(M\). (This is a point stressed by
\cite[Ch. 4]{Williamson1994-WILV}.) We cannot say that it is true to
degree 0.5 than (2) is true to degree 1, and true to degree 0.5 that it
is true to degree 0.8. For then it is only true to degree 0.5 that (2)
has some truth value or other. And the use of truth-tables to generate a
logic presupposes that every sentence has some truth values or other. If
this is not determinately true, \(M\) is not a complete theory. So the
model theory will show that our theory is substantially better than
\(M\) in this respect.

Consider the following (minor) variant on KT. Vary the syntax so
\(\square A\) is only well-formed if \(A\) is of the form
\(B \rightarrow C\). Call the resulting logic KT\textsubscript{R}, with
the R indicating the syntactic restriction. The restriction makes very
little difference. Since \(A\) is equivalent to
(\(A \rightarrow A\))~\(\rightarrow A\), even if \(\square A\) is not
well-formed in KT\textsubscript{R}, the KT-equivalent sentence
\(\square\)((\(A \rightarrow A\))~\(\rightarrow A\)) will be
well-formed. The Kripke models for KT\textsubscript{R} are quite
natural. \(\square\)(\(B \rightarrow C\)) is true at a point iff all
accessible points at which \(B\) is true are points at which \(C\) is
true. (There is no restriction on the accessibility relation other than
reflexivity.)

Since KT\textsubscript{R} is so similar to KT, we can derive most of its
formal properties by looking at the derivations of similar properties
for KT. (The next few paragraphs owe a lot to
\cite[Chs.1-3]{Goldblatt1992}.) Let's start with an axiomatic proof
theory. The axioms for KT\textsubscript{R} are:

\begin{itemize}
\item
  All classical tautologies
\item
  All well-formed instances of K:
  \(\square\)(\(A \rightarrow B\))~\(\rightarrow\)~(\(\square A \rightarrow \square B\))
\item
  All well-formed instances of T: \(\square A \rightarrow A\)
\end{itemize}

The rules for KT\textsubscript{R} are

\begin{description}
\item[Modus Ponens]
If \(A \rightarrow B\) is a theorem and \(A\) is a theorem, then \(B\)
is a theorem
\item[Restricted Necessitation]
If \(A\) is a theorem and \(\square A\) is well-formed, then
\(\square A\) is a theorem.
\end{description}

Given these, we can now define a maximal consistent set for
KT\textsubscript{R}. It is a set \(S\) of sentences with the following
three properties:

\begin{itemize}
\item
  All theorems of KT\textsubscript{R} are in \(S\).
\item
  For all \(A\), either \(A\) is in \emph{S} or \(\neg A\) is in \(S\).
\item
  \(S\) is closed under modus ponens.
\end{itemize}

The existence of Kripke models for KT\textsubscript{R} show that some
maximal consistent sets exist: the set of truths at any point will be a
maximal consistent set. The canonical model for KT\textsubscript{R} is
\(\langle W, R, V \rangle\) where

\begin{itemize}
\item
  \emph{W} is the set of maximal consistent sets for KT\textsubscript{R}
\item
  \emph{R} is a subset of \emph{W} \(\times\) \emph{W} such that
  \emph{w}\textsubscript{1}\emph{Rw}\textsubscript{2} iff for all \(A\)
  such that \(\square A \in\) \emph{w}\textsubscript{1},
  \(A \in\)~\emph{w}\textsubscript{2}
\item
  \(V\) is the valuation such that \(V\)(\(A\))~= \{\emph{w}:
  \(A \in\)~\emph{w}\}
\end{itemize}

Since all instances of T are theorems, it can be easily shown that
\emph{R} is reflexive, and hence that this is a frame for
KT\textsubscript{R} and hence that KT\textsubscript{R} is canonically
complete.

We can translate all sentences of KT\textsubscript{R} into a language
that contains \(\geqslant _T\) but not \(\square\). Just replace
\(\square\)(\(B \rightarrow A\)) with \(A \geqslant _T B\) wherever
\(\square\) occurs including inside sentences. (We appeal here and here
alone to the restriction in KT\textsubscript{R}.) Translating the axioms
for KT\textsubscript{R}, we get the following axioms for the logic of
\(\geqslant _T\).

\begin{itemize}
\item
  All classical tautologies
\item
  All instances of:
  (\(B\)~~\(\geqslant _T A\))~\(\rightarrow\)~((\(A \geqslant _T\)~(\(A \rightarrow A\)))~\(\rightarrow\)
  (\(B \geqslant _T\)~(\(B \rightarrow B\))))
\item
  All instances of:
  (\(B \geqslant _T A\))~\(\rightarrow\)~(\(A \rightarrow B\))
\end{itemize}

The rules are

\begin{description}
\item[Modus ponens]
If \(A \rightarrow B\) is a theorem and \(A\) is a theorem, then \(B\)
is a theorem.
\item[Determination]
If \(A \rightarrow B\) is a theorem, then \(B \geqslant _T A\) is a
theorem.
\end{description}

We can simplify somewhat by replacing the second axiom schema with

\begin{itemize}
\item
  All instances of:
  \(A \geqslant _T B \rightarrow\)~(\(B \geqslant _T C \rightarrow A \geqslant _T C\))
\end{itemize}

A Kripke model for this logic is just a Kripke model for KT, except we
say \(B \geqslant _T A\) is true at a point iff \(B\) is true at all
accessible points at which \(A\) is true. This leads to a semantic
definition of validity. An argument is valid iff it preserves truth at
any point in all such models.

Maximal consistent sets with respect to \(\geqslant _T\) and a canonical
model for \(\geqslant _T\) can be easily constructed by parallel with
the maximal consistent sets and canonical models for
KT\textsubscript{R}. These constructions show that if \(A\) is a theorem
of the logic for \(\geqslant _T\), then it is true at all points in all
models. More generally, they can be used to show that this logic is
canonically complete, though the details of the proof are omitted. The
maximal consistent sets for \(\geqslant _T\), i.e. the points in the
canonical model, just are the results of applying the translation rule
\(\square\)(\(B \rightarrow A\))~\(\Rightarrow A \geqslant _T B\) to the
(sentences in the) maximal consistent sets for KT\textsubscript{R}.

That's important because the points in the canonical model for
\(\geqslant _T\) are useful for understanding the relationship between
\emph{truer} and \emph{true}, and for understanding what languages are.
The set of true sentences in English is one of the points in the
canonical model for \(\geqslant _T\). For semantic purposes, languages
just are points in this canonical model. It is indeterminate just which
such point English is, but it is one of them. For many purposes it is
useful to think of the theory based on \emph{truer} as a variant on
\(M\). But considering the canonical model for \(\geqslant _T\)
highlights the similarities with supervaluationism rather than the
similarities with \(M\), for the points in the canonical model look a
lot like precisifications. It is, however, worth noting the many
differences between my theory and supervaluationism. I identify
languages with a single point rather than with a set of points, which
leads to the smoother treatment of higher-order vagueness on my account.
Also, I don't start with a set of acceptable points/precisifications.
The canonical model contains all the points that are formally
consistent, and I identify particular languages, like English, by
vaguely saying that the point that represents English is (roughly)
there. (Imagine my vaguely pointing at some part of the model when
saying this.) The most important difference is that I take the points,
with the truer than relation already defined, to be primitive, and the
accessibility/acceptability relation to be defined in terms of them.
This reflects the fact that I take the \emph{truer} relation to be
primitive, and determinacy to be defined in terms of it, whereas
typically supervaluationists do things the other way around. None of
these differences are huge, but they all favour my theory over
supervaluationism.

To return to the point about higher order vagueness, note that all of
the following sentences are consistent in KT, and hence their
`equivalents' using {\textgreater{}}\textsubscript{T} are also
consistent.

And obviously this pattern can be extended indefinitely. In general, any
claim of the form that \(a\) is an \(n\)-th order borderline case of an
\(F\) is consistent in this theory, as can be seen by comparison with
KT.

To close this section, I will note that we can also provide a fairly
straightforward natural deduction system for the logic of
\(\geqslant _T\). There are two philosophical benefits to doing this.
First, it proves my earlier claim that I can keep all inference rules of
classical logic. Second, it helps justify (A1) to (A10). Most rules
correspond directly to one of the constraints. For that reason I've set
all the rules, even though you've probably seen most of them before.

\begin{description}
\item[(\(\wedge\) In)]
\(\Gamma\) \(\vdash\) \(A\), \(\Delta\) \(\vdash\)
\(B \Rightarrow \Gamma \cup\Delta \vdash A \wedge B\)
\item[(\(\wedge\) Out-left)]
\(\Gamma\) \(\vdash\) \(A \wedge B \Rightarrow\) \(\Gamma\) \(\vdash\)
\(A\)
\item[(\(\wedge\) Out-right)]
\(\Gamma\) \(\vdash\) \(A \wedge B \Rightarrow\) \(\Gamma\) \(\vdash\)
\(B\)
\item[(\(\vee\) In-left)]
\(\Gamma\) \(\vdash\) \(B \Rightarrow\)~ \(\Gamma\) \(\vdash\)
\(A \vee B\)
\item[(\(\vee\) In-right)]
\(\Gamma\) \(\vdash\) \(A \Rightarrow\)~ \(\Gamma\) \(\vdash\)
\(A \vee B\)
\item[(\(\vee\) Out)]
\(\Gamma \cup\)\{\(A\)\} \(\vdash\) \(C\), \(\Delta \cup\)\{\(B\)\}
\(\vdash\) \(C\), \(\Lambda\) \(\vdash\) ~\(A \vee B \Rightarrow\)
\(\Gamma \cup \Delta \cup \Lambda \vdash C\)
\item[(\(\rightarrow\) In)]
\(\Gamma \cup\)\{\(A\)\} \(\vdash\) \(B \Rightarrow\)~ \(\Gamma\)
\(\vdash\) \(A \rightarrow B\)
\item[(\(\rightarrow\) Out)]
\(\Gamma\) \(\vdash\) \(A \rightarrow B\), \(\Delta\)
\(\vdash A \Rightarrow \Gamma \cup \Delta \vdash B\)
\item[(\(\neg\) In)]
\(\Gamma \cup\)\{\(A\)\} \(\vdash\) \(B \wedge \neg B \Rightarrow\)~
\(\Gamma\) \(\vdash\) \(\neg A\)
\item[(\(\neg\) Out)]
\(\Gamma\) \(\vdash \neg \neg A \Rightarrow\) \(\Gamma\) \(\vdash\)
\(A\)
\item[(\(\geqslant _T\) In)]
\(\Gamma\) \(\vdash\) \(A \Rightarrow\) \{\(B \geqslant _T C\):
\(B \in\)~ \(\Gamma\)\} \(\vdash\) ~\(A \geqslant _T C\)
\item[(\(\geqslant _T\) Convert)]
\(\Gamma\) \(\vdash\) ~\(A \geqslant _T B \Rightarrow\)~ \(\Gamma\)
\(\vdash\) (\(B \rightarrow A\))\(\geqslant _T C\)
\item[(\(\geqslant _T\) Out)]
\(\Gamma\) \(\vdash\) \(A \geqslant _T B \Rightarrow\)~ \(\Gamma\)
\(\vdash\) ~\(B \rightarrow A\)
\end{description}

(Thanks to Gabriel Uzquiano for several probing questions that led to
this section being written.)

\hypertarget{sexy-sorites}{%
\subsection{\texorpdfstring{Sexy Sorites
}{Sexy Sorites }}\label{sexy-sorites}}

A good theory of vagueness should tell us two things about the Sorites.
The easy part is to say what is wrong with Sorites arguments: not all
premises are perfectly true. The hard part is to say why the premises
looked plausible to start with. The \(M\) theorist has the beginnings of
a story, though not the end of a story. The beginning is that all the
premises in a typical Sorites argument are nearly true, and they look
plausible because we confuse near truth for truth. Can I say the same
thing, since my theory is like \(M\)? No, for two reasons. First, since
my theory explicitly gets rid of numerical representations of
intermediate truth values, I don't have any way to analyse \emph{almost
true}. Second, since I say that one of the Sorites premises is false,
I'd be committed to the odd view that some false sentence is almost
perfectly true. Thanks to Cian Dorr for pointing out this consequence.

The story the \(M\) theorist tells does not generalise. The problem is
that not all Sorites arguments involve conditionals. A typical Sorites
situation involves a chain from a definite \(F\) to a definite
not-\(F\). Let \(^\prime\) denote the successor relation in this
sequence, so if \(F\) is \emph{is tall} and \(a\) is 178cm tall, then
\(a ^\prime\) will be 177.99cm tall, assuming the sequence progresses
0.1mm at a time. According to \(M\), every premise like (SI) is almost
true.

\begin{description}
\item[(SI)]
If \(a\) is tall, then \(a ^\prime\) is tall.
\end{description}

But we could have built a Sorites argument with premises like (SA).

\begin{description}
\item[(SA)]
It is not the case that \(a\) is tall and \(a ^\prime\) is not tall.
\end{description}

And premises of this form are not, in general, almost true. Indeed, some
will have a truth value not much about 0.5. So \(M\) has no explanation
for why premises like (SA) look persuasive. This is quite bad, because
(SA) is \emph{more} plausible than (SI) as I'll now show. Consider the
following thought experiment. You are trying to get a group of
(typically non-responsive) undergraduates to appreciate the force of the
Sorites paradox. If they don't feel the force of (SI), how do you
persuade them? My first instinct is to appeal to something like (SA). If
that doesn't work, I appeal to theoretical considerations about how our
use of \emph{tall} couldn't possibly pick a boundary between \(a\) and
\(a ^\prime\). I think I find (SI) plausible \emph{because} I find (SA)
plausible, and I would try to get the students to feel likewise. There's
an asymmetry here. I wouldn't defend (SA) by appealing to (SI), and I
don't find (SA) plausible because it follows from (SI). (This is not to
endorse universally quantified versions of either (SA) or (SI). They are
like Axiom V - claims that remain intuitively plausible even when we
know they are false.)

Sadly, many theories have little to say about why (SA) seems true. The
official epistemicist story is that speakers only accept sentences that
are determinately, i.e. knowably, true. But some instances of (SA) are
actually \emph{false}, and many many more are not knowably true. The
supervaluationist story about (SA) is no better.

Here's a surprising fact about the Sorites that puts an unexpected
constraint on explanations of why (SA) is plausible. In the history of
debates about it, I don't think anyone has put forward a Sorites
argument where the major premises are like (SO).

\begin{description}
\item[(SO)]
Either \(a\) is not tall, or \(a ^\prime\) is tall.
\end{description}

(This point is also noticed in \citet{SiderBraun}.) There's a good
reason for this: (SO) is \emph{not} intuitively true, unless perhaps one
sees it as a roundabout way of saying (SA). In this respect it conflicts
quite sharply with (SA), which \emph{is} intuitively true. But hardly
any theory of vagueness (certainly not \(M\) or supervaluationism or
epistemicism) provide grounds for distinguishing (SA) from (SO), since
most theories of vagueness endorse DeMorgan's laws. Further, none of the
many and varied recent solutions to the Sorites that do not rely on
varying the underlying logic (e.g.
\citet{Fara2000, Sorensen2001, Eklund2002}) seem to do any better at
distinguishing (SA) from (SO). As far as I can tell none of these
theories \emph{could}, given their current conceptual resources, tell a
story about why (SA) is intuitively plausible that does not falsely
predict (SO) is intuitively plausible. That is, none of these theories
could solve the Sorites paradox with their current resources.

There is, however, a simple theory that does predict that (SA) will look
plausible while (SO) will not. Kit \citet{Fine1975a} noted that if we
assume that speakers systematically confuse \(p\) for
\emph{Determinately p}, even when \(p\) occurs as a constituent of
larger sentences rather than as a standalone sentence, then we can
explain why speakers may accept vague instances of the law of
non-contradiction, but not vague instances of the law of excluded
middle. (That speakers do have these differing reactions to the two laws
has been noted in a few places, most prominently \citet{Burgess1987} and
\citet{Tappenden1993}.) It's actually rather remarkable how many true
predictions one can make using Fine's hypothesis. It correctly predicts
that (5) should sound acceptable.

Now (5) is a contradiction, so both the fact that it sounds acceptable
if I am a borderline case of vagueness, and the fact that some theory
predicts this, are quite remarkable. This is about as good as it gets in
terms of evidence for a philosophical claim.

(We might wonder just why Fine's hypothesis is true. One idea is that
there really isn't any difference in truth value between \(p\) and
\emph{Determinately p}. This leads to the absurd position that some
contradictions, like (5), are literally true. I prefer the following
two-part explanation. The first part is that when one utters a simple
subject-predicate sentence, one implicates that the subject
\emph{determinately} satisfies the predicate. This is a much stronger
implicature than conversational implicature, since it is not
cancellable. And it does not seem to be a conventional implicature.
Rather, it falls into the category of nonconventional nonconversational
implicatures Grice suggests exists on pg. 41 of his
\citeyear{Grice1989}. The second part is that some implicatures,
including determinacy implicatures, are computed locally and the results
of the computations passed up to whatever system computes the intuitive
content of the whole sentence. This implies that constituents of
sentences can have implicatures. This theme has been studied quite a bit
recently; see \citet{Levinson2000} for a survey of the linguistic data
and \citet{Sedivy1999} for some empirical evidence supporting up this
claim. Just which, if any, implicatures are computed locally is a major
research question, but there is \emph{some} evidence that Fine's
hypothesis is the consequence of a relatively deep fact about linguistic
processing. This isn't essential to the current project - really all
that matters is that Fine's hypothesis is true - but it does suggest
some interesting further lines of research and connections to ongoing
research projects.)

If Fine's hypothesis is true, then we have a simple explanation for the
attractiveness of (SA). Speakers regularly confuse (SA) for (6), which
is true, while they confuse (SO) for (7), which is false.

This explanation cannot \emph{directly} explain why speakers find (SI)
attractive. My explanation for this, however, has already been given.
The intuitive force behind (SI) comes from the fact that it follows, or
at least appears to follow, from (SA), which looks practically
undeniable.

So Fine's hypothesis gives us an explanation of what's going on in
Sorites arguments that is available in principle to a wide variety of
theorists. Fine proposed it in part to defend a supervaluationist
theory, and \citet{Keefe2000} adopts it for a similar purpose. Patrick
\citet{Greenough2003} has recently adopted a similar looking proposal to
provide an epistemicist explanation of similar data. (Nothing in the
explanation of the attractiveness of Sorites premises turns on any
\emph{analysis} of determinacy, so the story can be told by
epistemicists and supervaluationists alike.) And the story can be added
to the theory of \emph{truer} sketched here. It might be regretted that
we don't have a \emph{distinctive} story about the Sorites in terms of
\emph{truer}. But the hypothesis that some sentences are truer than
others is basically a \emph{semantic} hypothesis, and if the reason
Sorites premises look attractive is anything like the reason (5) looks
\emph{prima facie} attractive, then that attractiveness should receive a
\emph{pragmatic} explanation. What is really important is that there be
some story about the Sorites we can tell.

\hypertarget{linearity-intuitions}{%
\subsection{Linearity Intuitions}\label{linearity-intuitions}}

The assumption that \emph{truer} is a non-linear relation is the basis
for most of the distinctive features of my theory, so it should be
defended. There are two reasons to believe it.

One is that we can't simultaneously accept all of the following five
principles.

\begin{itemize}
\item
  \emph{Truer} is a linear relation.
\item
  (A2), that conjunction is a greatest lower bound.
\item
  (A4), that negation is order inverting.
\item
  (A7), that contradictions are determinately false.
\item
  There are indeterminate sentences.
\end{itemize}

I think by far the least plausible of these is the first, so it must go.

Linearity (or at least determinate linearity) also makes it difficult to
tell a plausible story about higher order vagueness. Linearity is the
claim that for any two sentences \(A\) and \(B\), the following
disjunction holds. Either \(A\)~{\textgreater{}}\textsubscript{T}~\(B\),
or \(B\)~{\textgreater{}}\textsubscript{T}~\(A\), or \(A =_T B\). If
truer is determinately linear, that disjunction is determinately true.
And if \emph{truer} is linear, and if that disjunction is determinately
true, then one of its disjuncts must be determinately true, for
linearity rules out the possibility of a determinately true disjunction
with no determinately true disjunct. Now take a special case of that
disjunction, where \(B\) is 0=0. In that case we can rule out
\(A\)~{\textgreater{}}\textsubscript{T}~\(B\). So the only options are
\(B\)~{\textgreater{}}\textsubscript{T}~\(A\) or \(A =_T B\). We have
concluded that given linearity, one of these disjuncts must be
determinately true. That is, \(A\) is either determinately intermediate
or determinately determinate. But intuitively neither of these need be
true, for \(A\) might be in the `penumbra' between the determinately
intermediate and the determinately determinate. This argument is only a
problem if we assume determinate linearity, but it's hard to see the
theoretical motivation for believing in linearity but not determinate
linearity.

Still, it is very easy to believe in linearity. Even for comparatives
that are clearly non-linear, like \emph{more intelligent than}, there is
a strong temptation to treat them as linear. (Numerical measurements of
intelligence are obviously inappropriate given that \emph{more
intelligent than} is non-linear, but there's a large industry involved
in producing such measurements.) And this temptation leads to some prima
facie plausible objections to my theory. (All of these objections arose
in the discussion of the paper at BSPC.)

\hypertarget{true-and-truer-due-to-cian-dorr}{%
\paragraph*{True and Truer (due to Cian
Dorr)}\label{true-and-truer-due-to-cian-dorr}}
\addcontentsline{toc}{paragraph}{True and Truer (due to Cian Dorr)}

Here's an odd consequence of my theory plus the plausible assumption
that \emph{If S then S is true} is axiomatic. We can't infer from \(A\)
is true and \(B\) is false that \(A\) is truer than \(B\). But this
looks like a reasonably plausible inference.

If we added this as inference rule, we would rule out all intermediate
sentences. To prove this assume, for \emph{reductio}, that \(A\) is
intermediate. Since we keep classical logic, we know \(A \vee \neg A\)
is true. If \emph{A}, then \(A\) is true, and hence \(\neg A\) is false.
Then the new this inference rule implies \(A \geqslant _T \neg A\),
hence \(A\)\({\wedge}{\neg}\)\(A \geqslant _T \neg A\), since
\(\neg A \geqslant _T \neg A\), and hence 0=1\(\geqslant _T \neg A\),
since 0=1 \(\geqslant _T A \wedge \neg A\), and \(\geqslant _T\) is
transitive. So \(A\) is determinately true, not intermediate. A converse
proof shows that if \(\neg A\), then \(A\) is determinately false, not
intermediate. So by (\(\vee\)-Out) it follows that \(A\) is not
intermediate, but since \(A\) was arbitrary, there are no intermediate
truths. So this rule is unacceptable, despite its plausibility.

\hypertarget{comparing-negative-and-positive-due-to-jonathan-schaffer}{%
\paragraph*{Comparing Negative and Positive (due to Jonathan
Schaffer)}\label{comparing-negative-and-positive-due-to-jonathan-schaffer}}
\addcontentsline{toc}{paragraph}{Comparing Negative and Positive (due to
Jonathan Schaffer)}

Let \(a\) be a regular borderline case of genius, somewhere near the
middle of the penumbra. Let \emph{b} be someone who is not a determinate
case of genius, but is very close. Let \(A\) be \emph{a is a genius} and
\(B\) be \emph{b is a genius}. It seems plausible that
\(A \geqslant _T \neg B\), since \(a\) is right around the middle of the
borderline cases of genius, but \emph{b} is only a smidgen short of
clear genius. But since \emph{b} is closer to being a genius than \(a\),
we definitely have \(B \geqslant _T A\). By transitivity, it follows
that \(B \geqslant _T \neg B\), hence \(B\) is determinately true (by
the reasoning of the last paragraph). Since \(\neg B\) is not
determinately false, it follows that \(B \wedge \neg B\)~is not
determinately false, contradicting (A7).

Since I accept (A7) I must reject the initial assumption that
\(A \geqslant _T \neg B\). But it's worth noting that this case is quite
general. Similar reasoning could be used to show that for any
indeterminate propositions of the form \(x\) \emph{is a genius} and
\emph{y} \emph{is not a genius}, the first is not truer than the second.
This seems odd, since intuitively these could both be indeterminate
while the first is very nearly true and the second very nearly false.

\hypertarget{comparing-different-predicates-due-to-elizabeth-harman}{%
\paragraph*{Comparing Different Predicates (due to Elizabeth
Harman)}\label{comparing-different-predicates-due-to-elizabeth-harman}}
\addcontentsline{toc}{paragraph}{Comparing Different Predicates (due to
Elizabeth Harman)}

One intuitive way to understand the behaviour of \emph{truer} is that
\(A\) is truer than \(B\) iff \(A\) is true on every admissible
precisification on which \(B\) is true and the converse does not hold.
This can't be an analysis of \emph{truer}, since it assumes we can
independently define what is an admissible precisification, and this
seems impossible. But it's a useful heuristic. And reflecting on it
brings up a surprising consequence of my theory. If we assume that
precisifications of predicates from different subject areas (e.g.
\emph{hexagonal} and \emph{honest}) \emph{} are independent, it follows
that subject-predicate sentences involving those predicates and
indeterminate instances of them are incomparable with respect to truth.
But this seems implausible. If France is a borderline case of being
hexagonal that is close to the lower bound, and George Washington is a
borderline case of being honest who is close to the upper bound, then we
should think \emph{George Washington is honest} is truer than
\emph{France is hexagonal}.

All three of these objections seem to me to turn on an underlying
intuition that \emph{truer} should be a linear relation. If we are given
this, then the inference principle Dorr suggests looks unimpeachable,
and the comparisons Schaffer and Harman suggested look right. But once
we drop the idea that truer is linear, I think the plausibility of these
claims falls away. So the arguments against linearity are ipso facto
arguments that we should simply drop the intuitions Dorr, Schaffer and
Harman are relying upon.

To conclude, it's worth noting that a very similar inferential rule to
the rule Dorr suggests is admissible. From the fact that \(A\) is
determinately true, and \(B\) is determinately false, it follows that
\(A\) is truer than \emph{B}. If we assume, as seems reasonable, that
we're only in a position to \emph{say} that \(A\) is true when it is
determinately true, then whenever we're in a position to say \(A\) is
true and \(B\) is false, it will be true that \(A\) is truer than \(B\).
This line of defence is obviously similar to the explanation I gave in
the previous section of why Sorites premises look plausible, and to the
argument Rosanna Keefe gives that the failure of classical inference
rules is no difficulty for supervaluationism because it admits very
similar inference rules \citep{Keefe2000}.
