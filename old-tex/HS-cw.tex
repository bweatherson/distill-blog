\newcommand{\HS}{Hum\-e\-an Super\-ven\-i\-ence}

\section{What is \HS?}

As with many aspects of David Lewis's work, it is hard to provide a better summary of his views than he provided himself. So the following introduction to what the Humean Supervenience view is will follow the opening pages of \citet{Lewis1994a} extremely closely. But for those readers who haven't read that paper, here's the nickel version.

\HS\ is the conjunction of three theses.

\begin{enumerate}
\item \textbf{Truth supervenes on being} \citep{Bigelow1988}. That is, all the facts about a world supervene on facts about which individuals instantiate which fundamental properties and relations.
\item \textbf{Anti-haeccaetism}. All the facts about a world supervene on the distribution of qualitative properties and relations; rearranging which properties hang on which  `hooks' doesn't change any facts.
\item \textbf{Spatio-temporalism}. The only fundamental relations that are actually instantiated are spatio-temporal, and all fundamental properties are properties of points or point-sized occupants of points.
\end{enumerate}

\noindent The first clause is a core part of Lewis's metaphysics. It is part of what it is for some properties and relations to be fundamental that they characterize the world. Indeed, Lewis thinks something stronger, namely that the fundamental properties and relations characterize the world \textit{without redundancy} \citep[60]{Lewis1986a}. This probably isn't true, for a reason noted in \citet{SiderDiss}. Consider the relations \textit{earlier than} and \textit{later than}. If these are both fundamental, then there is some redundancy in the characterisation of the world in terms of fundamental properties and relations.  But there is no reason to believe  that one is fundamental and the other isn't. And it is hard to see how we could give a complete characterisation of the world without either of these relations. So we'll drop the claim that the fundamental properties relations characterise the world without redundancy, and stick to the weaker claim, namely that the fundamental properties and relations characterize the world completely.

The second clause is related to Lewis's counterpart theory. Consider what it would be like for anti-haeccaetism to fail. There would have to be two worlds, with the same distribution of qualitative properties, but with different facts obtaining in each. These facts would have to be non-qualitative facts, presumably facts about which individual plays which role. So perhaps, to use a well-known example, there could be a world in which everything is qualitatively as it is in this world, but in which Barack Obama plays the Julius Caeser role, and vice versa. So Obama conquers Gaul and crosses the Rubicon, Caeser is born in Hawai'i and becomes President of the United States. But what could make it the case that the Gaul-conqueror in that world is really \textit{Obama's} counterpart, and not Caeser's? Nothing qualitative, and nothing else it seems is available. So this pseudo-possibility is not really a possibility. And so on for all other counterexamples to anti-haeccaetism.

The third clause is the most striking. It says there are no fundamental relations beyond the spatio-temporal, or fundamental properties of extended objects. If we assume that `properties' of objects with parts are really relations between the parts, and anything extended has proper parts, then the second clause reduces to the first. I think it isn't unfair to read Lewis as holding both those theses.

Since for Lewis the fundamental qualities are all intrinsic, the upshot is that the world is characterized by a spatio-temporal distribution of intrinsic qualities. As Lewis acknowledged, this was considerably more plausible given older views about the nature of physics than it is now. We'll return to this point at great length below. But for now the key point to see the kind of picture \HS\ offers. The world is like a giant video monitor. The facts about a monitor's appearance supervene, plausibly, on intrinsic qualities of the pixels, plus facts about the spatial arrangement of the pixels. The world is 4-dimensional, not 2-dimensional like the monitor, but the underlying picture is the same.

\section{Supervenience}

Given the name \HS\, you might expect it to be possible to state \HS\ as a supervenience thesis. But this turns out to be hard to do. Here is one attempt at stating \HS\ as a supervenience thesis that is happily clear, and unhappily false.

\begin{description}
\item[Strong Modal \HS] For any two worlds where the spatio-temp\-oral distribution of fundamental qualities is the same, the contingent facts are the same.
\end{description}

\noindent But \HS\ does not make a claim this strong. It is consistent with \HS\ that there could be fundamental non-spatio-temporal relations. The only thing \HS\ claims is that no such relations are instantiated. In a pair of possible worlds where there are such relations, and the relations vary but the arrangement of qualities is the same, \textbf{Strong Modal \HS} will fail. In the Introduction to \citet{Lewis1986b}, he suggested the following weaker version.

\begin{description}
\item[Local Modal \HS] For any two worlds at which no alien properties or relations are instantiated, if the spatio-temporal distribution of fundamental qualities is the same at each world, the contingent facts are also the same.
\end{description}

\noindent An alien property(/relation) is a fundamental property(/relation) that is not actually instantiated. So this version of \HS\ says that to get a difference between two worlds, you have to either have a change in the spatio-temporal arrangement of qualities, or the instantiation of actually uninstantiated fundamental properties or relations.

But Lewis eventually decided that wouldn't do either. In response to \citet{Haslanger1994}, he conceded that enduring objects would generate counter\-examples to \textbf{Local Modal Hu\-m\-ean Supervenience} even if there were no alien properties or relations. So he fell back to the following, somewhat vaguely stated, thesis. (See \citet{Lewis1994a} for the concession, and \citet{Hall2010} for an argument that he should not have conceded this to Haslanger.)

\begin{description}
\item[Familiar Modal \HS] In any two ``worlds like ours'', if the spatio-temp\-oral distribution of fundamental qualities is the same at each world, the contingent facts are also the same  \citep[475]{Lewis1994a}.
\end{description}

\noindent What's a ``world like ours'? It isn't, I fear, entirely clear. But this doesn't matter for the precise statement of \HS. The three theses in section 1 are clear enough, and state what \HS\ is. The only difficulty is in stating it as a \textit{supervenience} thesis.

\section{What is Perfect Naturalness?}
That definintion does, however, require that we understand what it is for some properties and relations to be \textit{fundamental}, or, as Lewis put it following his discussion in \citet{Lewis1983e}, \textit{perfectly natural}. The perfectly natural properties and relations play a number of interconnected roles in Lewis's metaphysics and his broader philosophy.

Most generally, they characterise the difference between real change and `Cambridge change', and the related difference between real similarity, and mere sharing of grue-like attributes. This somewhat loose idea is turned, in \textit{Plurality}, into a definition of duplication.

\begin{quote}
\dots two things are duplicates iff (1) they have exactly the same perfectly natural properties, and (2) their parts can be put into correspondence in such a way that corresponding parts have exactly the same perfectly natural properties, and stand in the same perfectly natural relations. \citep[61]{Lewis1986a}
\end{quote}

\noindent The intrinsic properties are then defined as those that are shared between any two (possible) duplicates. So, as noted above, \HS\ says that the spatio-temporal distribution of intrinsic features of points characterises worlds like ours.

I've gone back and forth between describing these properties as fundamental and describing them as perfectly natural. And that's because for Lewis, the perfectly natural properties are in a key sense fundamental. For reasons to do with the nature of vectorial properties, I think this is probably wrong \citep{Weatherson2006-WEATAM}. That is, we need to  hold that some derivative properties are perfectly natural in order to get the definition of intrinsicness terms of perfect naturalness to work. But for Lewis, the perfectly natural properties and relations are all fundamental.

Part of what Lewis means by saying that some properties are fundamental is that all the facts about the world supervene on the distribution. (This is Bigelow's thesis that truth supervenes on being.) But I think he also means something stronger. The non-fundamental facts don't merely supervene on the fundamental facts; those non-fundamental facts are true \textit{because} the fundamental facts are true, and \textit{in virtue of} the truth of the fundamental facts.

The perfectly natural properties play many other roles in Lewis's philosophy besides these two. They play a key role in the theory of laws, for instance. They are a key part of Lewis's solution to the New Riddle of Induction \citep{Goodman1955}. And they play an important role in Lewis's theory of content, though just exactly what that role is is a matter of some dispute. (See \citet{Sider2001-SIDCOP} and \citet{Weatherson2003-WEAWGA} for one interpretation, and \citet{Schwarz2009} for a conflicting interpretation.)

Now it is a pretty open question whether any one division of properties can do all these roles. One way to solve the New Riddle (arguably Lewis's way, though this is a delicate question of interpretation) is to be a dogmatist (in the sense of \citet{Pryor2000}) about inductive projections involving a privileged class of properties. Lewis's discussion of the New Riddle at the end of \citet{Lewis1983e} sounds like he endorses this view, with the privileged class being the very same class as fundamentally determines the structure of the world, and makes for objective similarity and difference. But why should these classes be the same? It might make more sense to, for instance, endorse dogmatism about inductive projections of \textit{observational} properties, rather than about microphysical properties.

Lewis doesn't attempt to give a theoretically neutral definition of the perfectly natural properties. Rather, the notion of a perfectly natural property is introduced by the theoretical role it serves. But that theoretical role is very ambitious, covering many areas in metaphysics, epistemology and the theory of content. We might wonder whether claims like \HS\ have any content if it turns out nothing quite plays that theoretical role. I think there is still a clear thesis we can extract, relying on the connection between intrinsicness and naturalness. It consists of the following claims:

\begin{itemize}
\item There is a small class of properties and relations such that the contingent facts at any world supervene on the distribution of these properties and relations.
\item Each of these properties is an intrinsic property.
\item At the actual world, the only relations among these which are instantiated are spatio-temporal, and all the contingent facts supervene not merely on the distribution of fundamental qualities and relations, but also on the distribution of fundamental qualities and relations \textit{over points and point-sized occupants of points}.
\end{itemize}

\noindent Those theses are distinctively Lewisian, they are clearly entailed by \HS\ as Lewis's conceives of them, they are opposed in one way or another by those who take themselves to reject \HS, but they are free of any commitment to there being a single class of properties and relations that plays all the roles Lewis wants the perfectly natural properties and relations to play. So from now on, when I discuss the viability of \HS, I'll be discussing the viability of this package of views.

\section{Humean Supervenience and other Humean Theses}

Lewis endorsed many views that we might broadly describe as `Humean'. Of particular interest here are the following three.

\begin{itemize}
\item \textbf{\HS}.
\item \textbf{Nomological Reductionism}. Nomological properties and relations (including lawhood, chance and causation) are not among the fundamental properties and relations.
\item \textbf{Modal Combinatorialism}. Roughly, anything can co-exist with anything else.
\end{itemize}

\newcommand{\NR}{Nomological Reductionism}
\newcommand{\MC}{Modal Combinatorialism}

\noindent We've stated \MC\ extremely roughly, and will persist with using a fairly informal version of it throughout. For an excellent study of more careful versions of it, see \citet{Nolan1996-NOLRU}. But those details aren't as important to this debate. What is important for now is that all three of these theses are associated with what are known as Humean approaches to metaphysics in the contemporary literature. But how closely connected are they to each other, or for that matter to Hume.

One question about Humean Supervenience is just how it connects to the work of the historical Hume. This would be a little easier to answer if there was a broad scholarly consensus that Hume actually believed the kind of simple regularity thesis of causation that Lewis attributes to him at the start of \citet{Lewis1973b}. But it isn't clear that this is Hume's view \citep{Strawson2000}. What is true is that Hume was sceptical that we could know more about causation than that it was manifested in certain distinctive kinds of correlations. But it is a further step to say that Hume inferred that causation just consists of these distinctive kinds of correlations. 

A second question is how \HS, which perhaps should be referred to as so-called ``\HS'', or perhaps even better as ``Lewisian Supervenience'', relates to the kind of regularity theory that Lewis attributes to Hume, or to the prohibition on  necessary connections between distinct existences that underlies \MC.  Lewis seemed  to see the three theses as related.  Here he is explaining how he chose to name \HS\ (and recall that this \textit{isn't} backed up by any detailed exegesis of Hume).

\begin{quote}
\HS\  is named in honour of the great denier of necessary connections. It is the doctrine that all there is to the world is a vast mosaic of local matters of particular fact just one little thing and then another. \citep[ix]{Lewis1986b}
\end{quote}

\noindent  This is a slightly confusing passage, since it isn't clear why a violation of \HS\  would constitute a necessary connection of any kind. We will return to this point below. But it does seem to make clear that Lewis thought that \HS\ and \MC\  were connected, since \MC\ is much more closely connected to the denial that they can be necessary connections between distinct existences.

Compare how  Lewis introduces \HS\  when discussing the role of possible worlds in formulating trans-world  supervenience theses in \textit{Plurality}.

\begin{quote}
Are the laws, chances, and causal relationships nothing but patterns which supervene on this point-by-point distribution of properties? Cou\-ld two worlds differ in their wars without differing, somehow, somewhere, in local qualitative character? (I discuss this question of `\HS', inconclusively, in the Introduction to my \textit{Philosophical Papers}, volume II.) \citep[14]{Lewis1986a}
\end{quote}

\noindent  This seems to connect \HS\  closely to \NR,  since it makes the reducibility of the nomological  properties and relations central to the question of whether \HS\ is true. We can also, I think, see Lewis connecting \MC\ and \NR\  in a later passage in \textit{Plurality}  where he discusses why he doesn't believe that laws are necessary truths.

\begin{quote}
 Another use of [\MC]  is to settle -- or as opponents might say, to beg -- the question whether the laws of nature are strictly necessary. They are not \dots\  Episodes of bread-eating are possible because actual; as are episodes of starvation. Juxtaposed duplicates of the two, on the grounds that anything can follow anything; here is a possible world to violate the law bread nourishes. \dots\  It is no surprise that [\MC]  prohibited strictly necessary connections between distinct existences. What I have done is to take a Humean view about laws and causation, and use it instead as a thesis about possibility. Same thesis, different emphasis. \citep[91]{Lewis1986a}
\end{quote}

\noindent So for Lewis, these three theses  are meant to be closely connected.  And it is true that in the contemporary literature all three of them are frequently described as `Humean' theses. (Or at least they are so described in metaphysics and philosophy of science; again, we're bracketing questions of historical interpretation here.) But on second glance, it isn't as clear what the connection between the three theses could amount to. One immediate puzzle is that \HS\ is for Lewis a \textit{contingent} thesis, while the other two theses are necessary truths. The accounts of causation, lawhood and chance that he gives in defending \NR\ are clearly meant to hold in all kinds of worlds, not just worlds like ours. (Consider the amount of effort that is spent in \citet{Lewis2004a} at defending the theory of causation from examples involving wizards, action at a distance and so on.) And  the formulation of \MC\ in \textit{Plurality} leaves  little doubt that it is meant to be necessarily true.

This difference in modal status means that the theses can't be in any way equivalent. But you might think that they are in some way reinforcing. Even that isn't so clear. Consider the most dedicated kind of denier of \MC, namely the fatalist who thinks that every truth is a necessary truth. She will \textit{endorse} \HS. After all, she thinks that all the truths about the world supervene on any category of truths whatsoever, so they'll supervene on intrinsic properties of point-sized objects.

In the other direction, failures of \HS\ don't motivate compromising \MC. Imagine a world where occasionally there are pairs of people who can know what each other is thinking, even though there is no independent informational chain between the two of them. It is just that a telepathic connection exists. Moreover, there is no rhyme or reason to when a pair of people will be telepathic; it is simply the case that some pairs of people are. In such a world, it is plausible that \textit{being a telepathic pair} will be a fundamental relation. That's not a problem for \HS, since there aren't any such pairs in this world. But it does mean \HS\ is false in that world.

Assume that Daniels and O'Leary are a telepathic pair. Any duplication of the pair of them will also be telepathic, since by Lewis's preferred definition of duplication, duplication preserves all fundamental properties and relations. Does that mean there's a necessary connection between Daniels and O'Leary? Not really. The spirit of \MC\ is that you can duplicate any parts of any worlds, and combine them. One part of our world is Daniels. A duplicate of him need not include any telepathic connection to O'Leary; indeed, he has duplicates in worlds in which O'Leary is absent. Another part of the world is O'Leary; duplicates of him need not include a connection to Daniels. Putting the two together, there is a world where there are duplicates of Daniels and O'Leary, but no telepathic connection between the two. So \MC\ suggests that even when \HS\ fails, there won't be a necessary connection between distinct objects. So \HS\ really isn't that important to the idea that there are no necessary connection between distinct existences.

What's closer to the truth, I think, is that \HS\ is \textit{interesting} because of \MC. If \MC\ fails, then \HS\ doesn't capture anything important. In particular, it doesn't capture the idea that the nomic is somehow less fundamental than (some features of) the non-nomic. It is only given \MC\ that we can make these kinds of priority claims in modal terms.  Think about the philosopher who denies \MC\  on the grounds that laws of nature are necessarily true.  That philosopher will say that the laws supervene on the distribution of intrinsic properties of points,  because the laws supervene on any set of facts that you like. But they will deny that this makes the distribution of intrinsic properties of points more fundamental than the laws.  It is only given \MC\  that we can claim that supervenience theses are any guide whatsoever to fundamentality.

What about the connection between \NR\ and \HS? It can't be equivalence, since Lewis agrees that \HS\ fails in worlds in which \NR\ is true.  For the same reason, it can't be that failures of \HS\  entail failures of \NR.  What about the other direction? Could we imagine \NR\ failing while \HS\ holds?  I think this is a coherent possibility, but not at all an attractive one. (Compare, in this respect, the discussion of theories that ``qualify technically as Humean'' at \cite[485]{Lewis1994a}.) It requires that some of the irreducible, nomological properties be intrinsic properties of point-sized objects. Well, we could imagine two worlds where $F$ and $G$ are co-extensive, intrinsic properties of points, and in one of them it is a law that all $F$s are $G$s, and in the other it is a law that all $G$s are $F$s, and there are further intrinsic properties of all the points which are $F$ and $G$ which underlie these laws without making a difference to any of the other facts. So we imagine that the property \textit{being F in virtue of being G} is held by all these things in one world but not in the other, and this is a fundamental perfectly natural property. I don't think any of this is literally inconsistent, and I think filling out the details could give us a way for \NR\ to fail while the letter of \HS\ holds. But it would clearly violate the spirit of \HS\, and it isn't clear why we should believe in such `possibilities' anyway. 

So in practice, I think that any philosopher who rejects \NR\ is probably going to want to reject \HS. And I think that Lewis saw some of the deepest challenges to \HS\ as coming from threats to \NR.  In particular, Lewis  thought that the biggest challenges to \HS\  came from the difficulties in providing a reductive account of chance, and the appeal of non-reductive series of causation.

The difficulties in providing a reductive account of  chance are discussed at length in the introduction to \citet{Lewis1986b}, and  in the only paper that has `Hum\-e\-an Supervenience'  in its title, i.e., \citet{Lewis1994a}. Here is a quick version of the problem. Chances are not fundamental, so they must supervene on the distribution of qualities. At least in the very early stages of the universe, there aren't enough facts about the distribution of qualities in the past and present  to form a suitable subvenient base for the chances. So whether the chance of $p$ is $x$ or $y$  will, at least some of the time, depend on how the future of the world turns out. Now let $p$  the proposition that tells the full story about the future of the world. And assume that $p$ is a proposition such that what its chance is depends on how that future goes. If it goes the way $p$ says it will go, the chance of $p$ is $x$; if it goes some other way, the chance of $p$ is $y$.  Given a Humean  theory of chance, Lewis says that this is going to be possible.

\newcommand{\PP}{Principal Principle}
But now there's a problem. What Lewis calls the \PP\  says that if we know the chance of $p$ is $y$,  and have no further information, then our credence in $p$ should be $y$.  But in this case, if we knew the chance of $p$ was $y$,  we could be sure that $p$  would not obtain. So our credence in $p$  should be 0.  Here we seem to have reached a contradiction, and it is a contradiction to Lewis for a long time feared undermined the prospect of giving a reductive account of chance.  The solution he eventually settled on in \citet{Lewis1994a} was to slightly modify the \PP, with the modification being designed to make very little difference in regular cases, but avoid this contradiction.

Lewis discusses the appeal of non-reductive theories of causation in several places, most notably for our purposes \citet{Lewis2004a} and \citet{Lewis2004d}.  Much of his attention is focused on the theory developed by Peter \citet{Menzies1996}.  Menzies suggests that causation is the intrinsic relation that does the best job of satisfying folk platitudes about causation. A  consequence of Menzies's view  is that there is something that makes a difference to the intrinsic properties of pairs of causes and effects which doesn't supervene on either the intrinsic properties of the two ends of the causal chain, or on the spatio-temporal relations that hold between them.  This something will either be causation or will be something on which causation depends. Either way there is a problem for \HS,  since there will have to be a  perfectly natural relation  that is not spatio-temporal.

Lewis's response is to raise problems for the idea that causation could be an intrinsic relation.  One class of worries concerns the very idea that causation could be a relation. Lewis says that  absences can be causes and effects, but absences can't stand in any relations,  so causation must not be a relation. Another class of worries concerns the idea that causation could be intrinsic. Causation by double prevention, says Lewis, doesn't look  like it could be intrinsic. But intuitively there could be causation by double prevention. Yet another class of worries concerns the idea that causation could be a natural relation, or that there could be  any one thing that satisfies all the platitudes about causation. The vast array of different ways in which causes can bring about their effects in the actual world, he says, undermines this possibility.

Note that in both cases Lewis defends \HS\  simply by defending \NR. So I think it is fair to say that there's a close connection between the two in Lewis's overall theory.

\section{Why Care about \HS}

As is well-known, some surprising results in quantum mechanics  suggest that entanglement relations are somehow fundamental \citep{Maudlin1994}.  This suggests that \HS\  is actually false. If that's right, why should we care about philosophical arguments for \HS? Lewis's response to this challenge is somewhat disconcerting.

\begin{quote}
Really, what I uphold is not so much the truth of \HS\ as the \textit{tenability} of it. If physics itself were to teach me that it is false, I wouldn't grieve.

 That might happen: maybe the lesson of Bell's Theorem is exactly that \dots\  But I am not ready to take lessons in ontology from quantum physics as it now is. \dots\ If, after [quantum theory has been cleaned up],  it still teaches non-locality, I shall submit willingly to the best of authority.

What I want to fight are \textit{philosophical} arguments against \HS.  When philosophers claim that one or another commonplace feature of the world cannot supervene on the arrangement of qualities, I make it my business to resist. Being a commonsensical fellow  (except where unactualised possible worlds are concerned)  I will seldom deny that the features in question exist. I grant their existence, and do my best to show how they can, after all, supervene on the arrangement of qualities. \citep[xi]{Lewis1986b}
\end{quote}

\noindent  We can, I think, dismiss the point about quantum physics as it was in 1986.  The theory has been cleaned up in just the way Lewis wanted, and the claims about non-locality remain. Indeed, by the end of his life Lewis was willing to take lessons in ontology from quantum physics. See, for example, \citet{Lewis2004b}.  So what is at issue here is whether or not there are philosophical arguments against \HS.

But at this point we might wonder why we should care. If a theory is false, what does it matter whether its falsehood is shown by philosophy or by physics? We might compare the dismissive attitude Lewis takes towards Plantinga's attempts to show that reconstructions of the problem of evil as an argument do not rely solely on things provable in first-order logic \citep{Lewis1993b}. 

The answer I offered in \citet{Weatherson2009-WEADL} was that the philosophical defence of Hu\-m\-ean Supervenience was connected to the point of the last paragraph quoted above. Lewis wanted to save various features of our commonsensical picture of the world. And he wanted to do this without saying that philosophical reflection showed us that the picture of the world given to us by signs of somehow incomplete.  He wanted to defend what I called `compatibilism', something that I contrasted with eliminativism and expansionism. The eliminativists want to say that science shows us that some commonsensical feature of reality doesn't really exist. (See, for example, \citet{Churchland1981} for eliminativism about folk psychological states.) The expansionists want to say that since science (or at least physics) doesn't recognise certain features of reality, but they obviously exist, we need to posit that science (or at least physics) is incomplete. There are many stripes of philosophical expansionists, from theists to dualists to believers in agent causation.

Lewis wasn't averse in principle to either eliminativism or expansionism. One could, depending on exactly how one interpreted folk theory and science, classify him as an eliminativist about gods, and an expansionist about unactualised possible worlds. But his first tendency was always to support compatibilism. Compatibilists face what Frank \citet{Jackson1998} called the `location problem'. They have to show where the commonsensical features are located in the scientific picture. That is, they have to show how to reduce (in at least some sense of `reduce') or commonsensical concepts to scientific concepts. (Many compatibilists may bristle at the idea that they have to be reductionists; in recent decades the world has abounded with `non-reductive physicalists', who are precisely compatibilists in my sense, but who reject what they call `reductionism'. But as \citet{Lewis1994b} argued, these rejections often turn on reading too much into the notion of reduction. For that reason, Lewis would not have objected to being described as a reductionist about many everyday concepts.)

One way to perform such a reduction would be to wait until the best scientific theory is developed, and show where within it we find minds, meanings, morals and all the other exciting features of our ordinary worldview. But that could take a while, and philosophers could use something to do while waiting. In the meantime we could look for a recipe that should work no matter what physical theory the scientists settle on, or at least should work in a very wide range of cases. I think we can see Lewis's defence of \HS\ as providing such a recipe.

It is important to note here that Lewis's defence of \HS\ was largely \textit{constructive}. He didn't try to give a proof that there couldn't be more to the world than the arrangement of local qualities. At least, he didn't rest a huge amount of weight on such arguments. The arguments we will look at below for a functional construal of the nomological are, perhaps, hints at arguments of this type. But, in general, Lewis defended \HS\ by explicitly showing where the ordinary concepts fitted in to a sparse physical picture of reality, under the assumption that physics tells us that the world consists of nothing but a spatio-temporal arrangement of intrinsic qualities.

Now physics tells us no such thing. But it shouldn't matter. If the recipe Lewis provides works in the case of the `Humean' world, it should also work in the world physics tells us we actually live in.  The reduction of laws to facts about the distribution of fundamental qualities, and the reduction of chances and counterfactual dependencies to facts about laws,  and the reduction of causation to facts about chances and counterfactual dependencies, and the reduction of mind to facts about causation and the distribution of qualities, and the reduction of value to facts about minds, and so on  are all independent of whether physics tells us that we have to recognise relationships like entanglement as fundamental. In other words, if we can solve the location problem for the Humean world,  we can solve it for the actual world. And solving the location problem is crucial to defending compatibilism. And whether it is possible to defend compatibilism is a central concern of metaphysics.

I quoted above a passage from 1986 in which Lewis links \HS\ to compatibilism. It's worth noting that he returns to the point in 1994.

\begin{quote}
 The point of defending \HS\  is not to support react\-ionary physics, but rather to resist philosophical arguments that there are more things in heaven and earth in physics has dreamt of.  Therefore if I defend the \textit{philosophical}  tenability of \HS,  that defence can doubtless be adapted to whatever better supervenience thesis may emerge from better physics. \citep[474]{Lewis1994a}
\end{quote}

\noindent That is, the defence of \HS\ just is part of the argument against expansionism, and hence for compatibilism. That was the defence I offered in \citet{Weatherson2009-WEADL}  for the interest of Lewis's defence of \HS,  even if it were to turn out that \HS\ was refuted by physics.  I still think much of it is correct. In particular, I still think that Lewis wanted to defend compatibilism, and that the defence of \HS\  is key to the defence of \HS.  Indeed, I think there is pretty strong textual evidence that it was a major part of Lewis's motivation for defending \HS. But this explanation of why the defence of \HS\  is significant can't explain why Lewis was so worried about the failures of Humean theories of chance. After all, if all we are trying to do is show that science and commonsense are compatible, we could just take chances to be one of the fundamental features of reality given to us by science. There isn't any need, from the perspective of trying to reconcile science and common sense, to give a reductive account of chance. Yet Lewis clearly thought that giving a reductive account of chance was crucial to the defence of \HS. As he said,

\begin{quote}
There is one big bad bug: chance. It is here, and here alone, that I fear defeat. But if I'm beaten here, then the entire campaign goes kaput. \cite[xiv]{Lewis1986b}
\end{quote}

\noindent  I now think that attitude is very hard to explain if my earlier views about the significance of \HS\  are entirely correct. The natural conclusion is that there is something \textit{more} that the defence of \HS\  is supposed to accomplish. One plausible interpretation is that what it is supposed to accomplish is a vindication of the idea that the key nomological concepts are, in a sense, descriptive.  It's easiest to say what this sense is by contrasting it with the kind of view that Lewis rejected.

\newcommand{\HO}{H$_2$O}

We're all familiar with the standard story about `water'.  Our ordinary usage of the term latches onto some stuff in the physical world. That stuff is \HO. Some people think that's because our ordinary usage determines a property which \HO\ satisfies, others because we demonstratively pick out \HO\ in ordinary demonstrations of what it is we're talking about when we use the term `water'.  Either way,  we get to be talking about \HO\ when we use the word `water', even  if we are so ignorant of chemistry that we can't tell hydrogen and oxygen apart. Moreover, our term continues to pick out `water'  even in worlds that are completely free of hydrogen and oxygen, and even if such worlds have other stuff that plays a very similar functional role to the role water plays in the actual world.

Lewis was somewhat sceptical of this standard story about `water' \citep{Lewis2002b}. He thought that the ordinary term was ambiguous between our usage on which it picked out \HO,  and usage on which it picked out a role, a role that happens to be played by \HO\ in the actual world but which could be played by other substances in other worlds. But if  he thought the standard story about `water'  was at best, part right, he thought applying a similar story to `law', `cause' and `chance'  was wildly implausible.

If such a story were right, then we would expect to find worlds where there was some relation other than causation which played the causal role. Since the actual world is physical, any world in which nonphysical things stand in the kind of relations that causes and effects typically stand in should do. So, for instance, if we have a world where the castings of spells are frequently followed by transformations from human to toad form, we should have a world where spells don't cause such transformations but rather the spellcasting and the transformation stand in a kind of fool's cause relationship. But we see no such thing. In such magical worlds, spells cause transformations.

So whatever causation is, it doesn't look to be the kind of thing whose essence can be discovered by physics. Physics couldn't tell us anything about the essence of the relationship between the spell and the transformation into a toad. But, we think, physics can tell us a lot about the fundamental properties and relations are instantiated in the actual world. So causation must not be one of them.

Lewis has a number of other arguments against anti-descriptivist views about individual nomological concepts. These arguments strike me as rather strong in the case of lawhood and causation, and less strong in the case of chance.

If \textit{being F} and \textit{being F in virtue of a law} are both fundamental properties, then a plausible principle of modal recombination would suggest they could come apart. But they cannot; or at least they cannot in one direction. We want \textit{being F in virtue of a law} to entail \textit{being F}. That's easy if lawhood is defined in terms of fundamental properties of things; but it's hard to see how it could be if lawhood itself is fundamental \citep[xii]{Lewis1986b}.

A similar argument goes for causation. Assume that causation is a fundamental intrinsic relation that holds between things at different times. Consider, for instance, the causal relationship which holds between a throw of a rock (call it $t$) and the shattering of the window (call it $s$). As we noted above in the case of Daniels and O'Leary, several applications of \MC\ suggest that there will be a world just like this one in which $t$ is followed by $s$, but in which $t$ does not cause $s$. But such a world seems to be impossible. As we also noted above, such a view runs into trouble with causation by double prevention, which does not look to be intrinsic.

The last two paragraphs have been extremely quick arguments, but in both cases it seems to me that they can be tightened up so as to provide good arguments for some kind of descriptivist stance towards laws and causation. Chance is another matter.

The first problem is that recombination arguments if anything point \textit{away} from descriptivism about chance. Any such account will imply that chances can't, in general, point too far away from frequencies. But recombination arguments suggest that chances and frequencies can come arbitrarily far apart. Consider some particular event type $e$ that has a one-half chance of occurring in circumstances $c$. Start with a world where $c$ occurs frequently, and about half the time it is followed by $e$. Now use recombination to generate a world where all the $c \wedge \neg e$ events are deleted, so $c$ is always followed by $e$. Unless we add a lot of bells and whistles to our theory of chance, it will no longer be the case that the chance of $e$ given $c$ is one-half. That is odd; we can't simply take the first circumstance where $c$ occurred and at that moment there was a one-half chance of it being followed by $e$, and patch it into an arbitrary world. \citet{BigelowCollinsPargetter} turn this idea into a more careful argument against descriptivism about chance. They say that chances should satisfy the following principle. (In this principle, $Ch$ is the chance function, and various subscripts relativise it to times and worlds.)

\begin{quote}
Suppose $x > 0$ and $Ch_{tw}(A) = x$. Then $A$ is true in at least one of those worlds $w^{\prime}$ that matches $w$ up to time $t$ and for which $Ch_t(A) = x$. \citep[459]{BigelowCollinsPargetter}
\end{quote}

\noindent That is, if the chance of $A$ at $t$ is $x$, and $x > 0$, then $A$ could occur without changing the history prior to $t$, and without changing the chance of $A$ at $t$. This seems like a plausible principle of chance, but it entails the not-so-Humean view that chances at $t$ supervene on history to $t$, not on the full state of the world.

Now as it turns out Lewis \textit{doesn't} rest on recombination arguments against rival views of chance, and in my view he is wise to do so. Instead he rests on epistemological arguments. He takes the following two things to be data points.

\begin{enumerate}
\item Something like the Principal Principle is true. The original Principal Principle said that if you knew the chance of $p$ at $t$ was $x$, and didn't have any `inadmissible' information (roughly, information about how the world developed after $t$), then your credence in $p$ should be $x$. Lewis tinkered with this slightly, as we noted above, but he took it to be a requirement on a theory of chance that the Principal Principle turn out at least roughly right.
\item The correct theory of chance will \textit{explain} the Principal Principle.
\end{enumerate}

\noindent Lewis frequently wielded this second requirement against rival theories of chance. Here's one example.

\begin{quote}
I can see, dimly, how it might be rational to conform my credences about outcomes to my credencs about history, symmetries and frequencies. I haven't the faintest notion how it might be rational to conform my credences about outcomes to my credences about some mysterious unHumean magnitude. Don't try to take away the mystery my saying that this unHumean magnitude is none other than \textit{chance}! \citep[xv]{Lewis1986b}
\end{quote}

\noindent But this also seems like a weak argument. For one thing, chances are actually correlated very well with frequencies, and this correlation does not look at all accidental. It seems very plausible to me that we should line up our credences with things that are actually correlated well with frequencies. But, you might protest, shouldn't we have an explanation of why the Principal Principle is an \textit{a priori} principle of rationality? I think that before we ask for such an explanation, we should check how confident we are that the Principal Principle, or anything else, is part of an \textit{a priori} theory of rationality. I'm not so confident that we'll be able to do this \citep{WeathersonSRE, Weatherson2007}.

There are other replies too that we might make. It seems plausible that we should minimise the expected inaccuracy of our credences \citep{Joyce1998}. This is true when we consider not just the \textit{subjective} expected inaccuracy of our credences, but the \textit{objective} expected inaccuracy of our credences. That is, when we calculate the expected inaccuracy of someone's credences, using chances as the probabilities for generating the expectations, it is good if this expected inaccuracy is as low as possible. But, assuming that we are using a proper scoring rule for measuring the accuracy of credences, this means that we must have credences match chances.

More generally, I'm very sceptical of theories that insist our metaphysics be designed to have complicated epistemological theses fall out as immediate consequences. Rationality requires that we be inductivists. Why is that? Here's a bad way to go about answering it: find a theory of persistence that makes induction obviously rational, and then require our metaphysics to conform to that theory. I don't think you'll get a very good theory of persistence that way, and, relatedly, you won't get a very Lewisian theory of persistence that way. The demand that the theory of chance play a central role in an explanation of the Principal Principle strikes me as equally mistaken.

If what I've been saying so far is correct, then chance interacts with the motivation for \HS\ in very different ways to how laws and causation interact. Neither of the two kinds of motivations for defending \HS\ against philosophical attacks provides us with good reason to leave chances out of the subvenient base on which we say all contingent facts supervene. This is not to yet offer anything like a positive argument for chances to be part of the fundamental furniture of reality. Rather, what I've argued here is that a metaphysics that takes chances as primitives would not be as far removed from a recognisably Lewisian metaphysics as a metaphysics that takes laws or causes as primitive, let alone one that takes mind, meanings or morals as primitive.

\section{Points, Vectors and Lewis}
The other main point from the discussion of the previous section is that the fact that quantum mechanics raises problems for \HS\  does not undercut the philosophical significance of Lewis's defence of \HS. But is \HS\ even  compatible with classical physics? Perhaps not.

\begin{quote}
Even classical electromagnetism raises a question for \HS\  as I stated it. Denis \citet{Robinson1989}  has asked: is a vector field an arrangement of local qualities? I said qualities were intrinsic; that means they can never differ between duplicates; and I would have said offhand that two things can be duplicates even if they point in different directions.  May be this last opinion should be reconsidered, so that vector-valued magnitudes may count as intrinsic properties. What else could they be? Any attempt to reconstruct with them as relational properties seems seriously artificial. \citep[474]{Lewis1994a}
\end{quote}

\noindent  The opinion that the Lewis proposes to discard here seems more than an offhand judgement. It seems to follow from the very way that we introduce the notion of duplication. Here is Lewis's own  attempt to introduce the notion.

\begin{quote}
We are familiar with cases of approximate duplication, e.g., when we use copying machines. And we understand that if these machines were more perfect than they are, the copies they made would be perfect duplicates of the original. Copy and original would be alike in size and shape and chemical composition of the ink marks and the paper, alike in temperature and magnetic alignment and electrostatic charge, alike even in the exact arrangement of their electrons and quarks. Such duplicates would be exactly alike we say. They would match perfectly, they would be qualitatively identical, they would be indiscernible. \citep[355]{Lewis1983e}
\end{quote}

\noindent  If Lewis is right that vector-valued magnitudes may count as intrinsic properties, then there is yet another condition that the perfect copying machine must satisfy. The original and the duplicate must be parallel. This isn't the case in most actual copying machines. Usually, the original is laid flat, while the duplicate is a small angle to make it easier to collect. This is a feature, not a bug. It is not a way in which the machine falls short of perfect copying. But if vector-valued magnitudes are intrinsic qualities, and duplicates share their intrinsic qualities, it would be. So Lewis is wrong to think that these vector-valued magnitudes may be intrinsic.

Moreover, the little argument that Lewis gives seems to rest on a category mistake. What matters here is the division of properties into intrinsic and extrinsic. But the properties on the kind of things that can be relational or non-relational.  As \citet{Humberstone1996}  shows, concepts and not properties of the things that can be relational and non-relational. For instance the concept \textit{being the same shape as David Lewis actually was at noon on January 1, 1970}, is a relational concept that presumably picks out an intrinsic property, namely a shape property. Whether they are valued magnitudes are intrinsic or extrinsic properties, is somewhat orthogonal question of whether it is best to pick them out by means of relational or non-relational concepts.

 There is a further issue about the compatibility of \HS\  with classical physics. This is a point that has been made well by Jeremy \citet{Butterfield2006},  and we can see the problem by looking at the different ways in which Lewis introduces \HS. 

\begin{quote}
\HS\ says that in a world like ours, the fundamental properties are local qualities: perfectly natural intrinsic properties of points, or of point-sized occupants of points. \citep[474]{Lewis1994a}
\end{quote}

\noindent Lewis goes back and forth between local properties and intrinsic properties of points here. These aren't the same thing. As Butterfield notes, `local'  is used in a few different ways throughout physics.  One simple usage identifies local properties of a point with properties that supervene on intrinsic features of arbitrarily small regions around the point. To take an important example, the slope of a curve at a point may be a local property of the curve at that point without being intrinsic property of the point.

This raises a question: can we do classical physics with only  intrinsic properties of points, and not even these further local properties? Butterfield argues, persuasively, that the answer is no. He notes, however, that there are some very mild weakenings of \HS\  that avoid this difficulty. Here is a very simple one.

Call \textbf{Local Supervenience} the following thesis. For any length $\varepsilon$ greater than 0, there is a length $d$ less than $\varepsilon$ with the following feature. All the facts about the world supervene on intrinsic features of objects and regions with diameter at most $d$, plus facts about the spatio-temporal arrangement of these objects and regions. This will mean that we can include all local qualities in the subvenient base, without assuming that these are intrinsic qualities of points. If the theory of intrinsicness in \citet{Weatherson2006-WEATAM} is correct, we'll also be able to include vector-valued magnitudes in the subvenient base without assuming that these are intrinsic properties of points. (On my view, they will end up being intrinsic properties of asymmetrically shaped regions.) We still won't be able to accommodate entanglement relationships, but we will be able to capture classical physics. And, for the reasons discussed in the previous section, it would still be worthwhile to ask whether there are philosophical objections to Local Supervenience. A negative answer would greatly assist the arguments for compatibilism, and for nomological descriptivism.

 Butterfield offers from theses like Local Supervenience to Lewis as friendly suggestions. But he thinks Lewis's focus on  points and their properties would have led him to reject it. I don't want to get into the business of making counterfactual speculation about what Lewis would or would not have accepted. But I think he should have been happy to weaken \HS\ to something like Local Supervenience. If the point of defending \HS\ is not to defend its truth, but rather to assist in larger arguments for compatibilism, and for nomological descriptivism, then the big question to ask is whether a defence of Local Supervenience (against distinctively philosophical objections)  would have served those causes just as well.  And I think it's pretty clear that it would have. Showing that we have no philosophical reason to posit fundamental non-local features of reality would be enough to let us ``resist philosophical arguments that there are more things in heaven and earth in physics has dreamt of'' \citep[474]{Lewis1994a}. Lewis's work in defending \HS\ has been invaluable to those of us who want to join this resistance. It wouldn't have been undermined if he'd allowed some local properties into the mix.
