

%
%
%
%
%
%%%%%%%%%%%%%%%%%%
%Questioning Contextualism
%%%%%%%%%%%%%%%%%%
%
%
%
%
%
\chapter{Questioning Contextualism}

\pubdata{ \textit{Aspects of Knowing}, edited by Stephen Hetherington, Elsevier, 2006, pp. 133-147. Thanks to David Chalmers, Keith DeRose, Tamar Szab\'{o} Gendler, Stephen Hetherington, John MacFarlane, Ishani Maitra and Matthew Weiner for helpful discussions about earlier drafts.}

There are currently a dizzying variety of theories on the market holding that whether an utterance of the form \textit{A knows that p} is true depends on pragmatic or contextual factors. Even if we allow that pragmatics matters, there are three questions to be answered. First, \textit{whose} interests matter? Here there are three options: the interests of \textit{A} matter, the interests of the person making the knowledge ascription matter, or the interests of the person evaluating the ascription matter. Second, \textit{which} kind of pragmatic factors matter? Broadly speaking, the debate here is about whether practical interests (the stakes involved) or intellectual interests (which propositions are being considered) are most important. Third, \textit{how} do pragmatic factors matter? Here there is not even a consensus about what the options are.

This paper is about the first question. I'm going to present some data from the behaviour of questions about who knows what that show it is not the interests of the person making the knowledge ascription that matter. This means the view normally known as \textit{contextualism} about knowledge-ascriptions is false. Since that term is a little contested, and for some suggests merely the view that someone's context matters, I'll introduce three different terms for the three answers to the first question. 

Consider a token utterance by B of \textit{A knows that p}. This utterance is being evaluated by C. A \textit{semantic pragmatist} about knowledge ascriptions says that whether C can correctly evaluate that utterance as true or false depends on some salient party's context or interests.\footnote{Note that the focus here is on the truth or falsity of the utterance, and not on the truth or falsity of the proposition the utterance expresses. I'm indebted to John MacFarlane for stressing to me the importance of this distinction.} We can produce a quick taxonomy of semantic pragmatist positions by looking at which of the three parties is the salient one.

\begin{description}
\item[Type A pragmatism] A's context or interests matter.
\item[Type B pragmatism] B's context or interests matter.
\item[Type C pragmatism] C's context or interests matter.
\end{description}

\noindent Type A pragmatism is defended by \citet{Hawthorne2004} under the name `subject\hyp{}sensitive invariantism' and \citet{Stanley2005-STAKAP} under the name `interest-relative invariantism'. The theory commonly known as \textit{contextualism}, defended by \citet{Cohen1986}, \citet{DeRose1995} and \citet{Lewis1996b}, is a form of \textit{Type B} pragmatism. On their theory, the semantic value of `knows' depends on the context in which it is uttered, i.e. B's context. (And the salient feature of B's context is, broadly speaking, B's interests.) Recently \citet{MacFarlane2009-MACNC} has outlined a very different variant of Type B pragmatism that we will discuss below. Type C pragmatism is the radical view that the token utterance does not have a context-invariant truth value, so one and the same utterance can be properly evaluated as true by one evaluator and false by another. This position is outlined, and defended, by \citet{MacFarlane2005-Knowledge}.

The purpose of this paper is to argue \textit{against} Type B pragmatism. I'll show that there is a striking disanalogy between the behaviour of `knows' in questions and the behaviour of terms for which a Type B-type theory is true. The best explanation for this disanalogy is that Type B theories, contextualism included, are false.\footnote{My own view is that if any pragmatic theory is true, it is Type A pragmatism. In \citet{Weatherson2005-WEACWD} I defend the view that whether an agent is sufficiently confident in \textit{p} to count as believing that \textit{p} depends on features of her context. In contexts where it is very important for practical deliberation whether \textit{p} is true, a degree of confidence that might ordinarily count as belief that \textit{p} might no longer so count. This is an odd kind of doxastic externalism, namely the view that whether a state amounts to a \textit{belief} is environment-dependent. Now to know that \textit{p} also requires having a certain level of confidence that \textit{p} is true, and it is arguable (though I haven't argued it) that this level is also dependent on the would be knowers environment. It is also arguable that the best theory of epistemic \textit{defeaters} will contain pragmatic features.\par Though I describe Type C pragmatism as radical, I've been convinced by John MacFarlane's work that it is a perfectly coherent doctrine. In \citet{Egan2005-EGAEMI} we defend a version of Type C pragmatism for sentences of the form \textit{A might be F}, where \textit{might} is understood epistemically. }

I won't be addressing the second or third questions here, but I'll assume (basically for ease of exposition) that the right answer to the second has more to do with \textit{practical} than \textit{intellectual} interests. So a sceptical possibility becomes relevant not because anyone is actively considering it, but because it makes a difference to someone's actions. Everything I say here should be restatable given any answer to that question, so little beyond exposition turns on this.

\section{Basic Indexicals in Questions}

As well as considering these three pragmatic theories about `know', we can consider similar theories about other words. We'll start with words that are universally considered to be indexicals. Consider the following three theories about `here'.

\begin{description}
\item [Type A pragmatism] A token of `here' denotes the location of the subject of the sentence in which it appears.
\item [Type B pragmatism] A token of `here' denotes the location of the speaker.
\item [Type C pragmatism] A token of `here' denotes the location of the person evaluating the sentence.
\end{description}

\noindent So when C evaluates B's utterance \textit{A is here}, the Type A pragmatist says that it gets evaluated as true iff A is where A is, the Type B pragmatist says that it gets evaluated as true iff A is where B is, and the Type C pragmatist says that it gets evaluated as true iff A is where C is. Obviously Type B pragmatism is true in general about `here', as it is for all obvious indexicals.\footnote{Whether the `in general' is strictly necessary turns on tricky considerations about non-standard usages, such as answering machines. I'm generally favourable to the view that if any qualification is needed it is a very small one. See \citet{Weatherson2002-WEAMI} for more discussion of this point.} What we're interested in for now, however, is not its truth but the kind of evidence that can be adduced for it. One very simple way of separating out these three theories involves questions, as in the following example.

Watson is at a party in Liverpool, and he is worried that Moriarty is there as well. He calls Holmes, who is tracking Moriarty's movements by satellite, and asks (1).

\numbex{0}{
\item Is Moriarty here?
}

\noindent This question has three properties that are distinctive of questions involving indexicals.

\begin{description}
\item [SPEAKER] How Holmes should answer the question depends on the speaker's (i.e. Watson's) environment, not his own and not Moriarty's. It would be wrong to say ``Yes'' because Moriarty is where he is, or ``No'' because Moriarty is not in the lab with Holmes following the satellite movement.
\item [CLARIFICATION] It is permissible in certain contexts to ask the speaker for more information about their context before answering. If Holmes's is unsure of Watson's location, he can reply with ``Where are you?'' rather than answering directly.\footnote{It is important here that we restrict attention to clarificatory questions about the context, and in particular to features of the context that (allegedly) affect the truth value of utterances involving the indexical. Speakers can always ask for clarification of all sorts of features of a question, but a question only has CLARIFICATION if it is appropriate to ask for clarifying information about the nature of the questioner's context. }
\item [DIFFERENT ANSWERS] If two different people, who are in different locations, ask Holmes (1), he can answer differently. Assume that Moriarty is at the party, so Holmes says ``Yes'' in reply to Watson. Lestrade then calls from Scotland Yard, because he is worried that Moriarty has broken in and asks Holmes (1). Holmes says ``No'', and this is consistent with his previous answer.
\end{description}

\noindent Questions involving indexicals have a fourth property that we'll return to from time to time in what follows.

\begin{description}
\item [NEGATIVE AGREEMENT] It is coherent to answer the question by saying ``No'', then basically repeating the question, inverting the verb-subject order so you have an indicative sentence. So Holmes could consistently (if falsely in this story) say, ``No, Moriarty is here.''
\end{description}

\noindent The truth of Type B Pragmatism about `here' explains, indeed predicts, that (1) has these properties. If a correct answer is a true answer, then Type B Pragmatism directly entails that (1) has SPEAKER. That assumption (that correctness is truth) plus the fact that you can speak to someone without knowing their location implies that (1) has CLARIFICATION, and adding the fact that different people can be in different locations implies that (1) has DIFFERENT ANSWERS. Whether Holmes can say \textit{Moriarty is here} depends whether he can truthfully answer his own question \textit{Is Moriarty here?} so NEGATIVE AGREEMENT is related to DIFFERENT ANSWERS. For these reasons it isn't surprising that other terms that are agreed on all sides to be indexicals generate questions that have all four properties. For instance, because `me' is an indexical, (2) has all four properties.

\numbex{1}{
\item Does Moriarty want to kill me?
}

\noindent This suggests a hypothesis, that all terms for which Type B pragmatism is true generate questions with those four properties. This hypothesis seems to be false; some terms for which Type B pragmatism is true do not generate questions that have NEGATIVE AGREEMENT. But a weaker claim, that all terms for which Type B pragmatism is true generate questions with the first three properties, does seem to be true.\footnote{The failure of NEGATIVE AGREEMENT for questions involving quantifiers, comparative adjectives and similar terms is interesting is interesting, and is something that a full semantic theory should account for. My feeling is that the best explanation will draw on the resources of a theory like that defended by \citet{Stanley2000-STACAL, Stanley2002-STANR} but it would take us far away from our primary purpose to explore this here. It is certainly an argument in favour of the `semantic minimalism' defended by \citet{Cappelen2005} that all and only the terms in their `basic set' (apart from tense markers), i.e. the obvious indexicals, generate questions with NEGATIVE AGREEMENT. I think the fact that many terms generate questions with the SPEAKER, CLARIFICATION and DIFFERENT ANSWERS property tells more strongly against their view.} Or so I will argue in the next section.

\section{Other Type B terms in Questions}

Various philosophers have argued that a version of Type B pragmatism is true for each of the following four terms: `tall', `empty', `ready' and `everyone'. Various defenders of Type B pragmatism about `knows' have argued that `knows' is analogous to terms on this list. In this section I'll argue that questions involving those four terms have the first three properties of questions listed above, though they don't seem to have NEGATIVE AGREEMENT. In the next section I'll argue that questions involving `know' do not have those three properties. These facts combine to form an argument against the hypothesis that Type B pragmatism is true of `knows'. (The argument here does not turn on supposing that Type B pragmatism is true of these four terms. All I want to argue for is the claim that any term for which Type B pragmatism is true generates questions with the first three properties. It is possible that other terms also generate these kinds of questions, but this possibility doesn't threaten the argument.) The examples involving these four terms will be a little complicated, but the intuitions about them are clear.

Moriarty has hired a new lackey, a twelve year old jockey. She is tall for a twelve year old girl, and tall for a jockey, but not tall for a person. Moriarty is most interested in her abilities as a jockey, so he worries that she's tall. Holmes is also most interested in her \textit{qua }jockey. Watson has noted that Moriarty never hires people who are tall for adults (he thinks this is because Moriarty likes lording over his lackeys) and is wondering whether the new hire fits this property. He asks Holmes (3).

\numbex{2}{
\item Is Moriarty's new lackey tall?
}

\noindent Holmes should say ``No''. What matters is whether she is tall \textit{by the standards Watson cares about}, not whether she is tall by the standards Holmes cares about (i.e. jockeys) or the standards Moriarty cares about (i.e. also jockeys). So (3) has SPEAKER. Lestrade wants information from Holmes about the lackey so his men can pick her up. They have this conversation.

\begin{quote}
Lestrade: Is Moriarty's new lackey tall? My men are looking for her.

Holmes: Where are they looking?

Lestrade: At her school.

Holmes: Yes, she looks like she could be fourteen or fifteen.
\end{quote}

\noindent Holmes quite properly asks for a clarification of the question so as to work out what standards for tallness are in play, so (3) has CLARIFICATION. And he properly gives different answers to Watson and Lestrade, so it has DIFFERENT ANSWERS. But note that `tall' doesn't have NEGATIVE AGREEMENT. Holmes can't answer (3) with ``No, she's tall.'' I won't repeat this observation for the other terms discussed in this section, but the point generalises.

Next we'll consider `empty'. Holmes and Watson are stalking Moriarty at a party. Watson is mixing poisons, and Holmes is trying to slip the poison into Moriarty's glass. Unfortunately Moriarty has just about finished his drink, and might be about to abandon it. Watson is absent-mindedly trying to concoct the next poison, but he seems to have run out of mixing dishes.

\begin{quote}
Watson: Is Moriarty's glass empty?

Holmes: What do you want it for?

Watson: I need something dry to mix this poison in.

Holmes: No, it's got a small bit of ice left in it.

(Lestrade arrives, and sees Holmes holding a vial.)

Lestrade: Why haven't you moved in? Is Moriarty's glass empty?

Holmes: Yes. He should get another soon.
\end{quote}

\noindent Holmes behaves entirely appropriately here, and his three responses show that the question \textit{Is Moriarty's glass empty?} has the CLARIFICATION, SPEAKER and DIFFERENT ANSWERS properties respectively. Note in particular that the glass is empty by the standards that matter to Moriarty and Holmes (i.e. it's got not much more than ice left in it) doesn't matter to how Holmes should answer until someone with the same interests, Lestrade, asks about the glass.

Third, we'll look at `ready'. Moriarty is planning four things: to rob the Bank of England, to invade Baker St and kill Holmes, to invade Scotland Yard to free his friends, and to leave for a meeting of criminals where they will plan for the three missions. Moriarty cares most about the first plan, Holmes about the second, and Lestrade about the third, but right now Watson cares most about the fourth because it's his job to track Moriarty to the meeting. Holmes is tracking Moriarty through an installed spycam.

\begin{quote}
Watson: Is Moriarty ready?

Holmes: Yes. You should go now.

(Watson departs and Lestrade arrives)

Lestrade: Is Moriarty ready?

Holmes: Who's that? Oh, hello Inspector. No, he still has to plan the attack out.
\end{quote}

\noindent Again, Holmes's answers show that the question \textit{Is Moriarty ready?} has the CLARIFICATION, SPEAKER and DIFFERENT ANSWERS properties. Note that in this case the issue of whether Moriarty is ready for the thing he cares most about, and the issue of whether he is ready for the thing Holmes cares most about, are not relevant to Holmes's answers. It is the interests of the different speakers that matter, which suggests that if one of the three types of pragmatism is true about `ready', it is Type B pragmatism.

Finally we'll look at `everyone'. \citet{Lewis1996b} suggests that `know' is directly analogous to `every', so the two words should behave the same way in questions. Moriarty's gang just robbed a department store while the royal family, along with many police, were there. Holmes is most interested in how this affected the royals, Watson in how the public reacted, Lestrade in how his police reacted, and Moriarty merely in his men and his gold. The public, and the royals, were terrified by the raid on the store, but the police reacted bravely.

\begin{quote}
Watson: Did Moriarty's men terrify everyone? 

Holmes: Her majesty and her party were quite shocked. Oh, you mean \textit{everyone}. Yes, the masses there were completely stunned.

(Lestrade enters.)

Lestrade: I just heard about the raid. How did they get through security? Did Moriarty's men terrify everyone?

Holmes: No, your men did their job, but they were outnumbered.
\end{quote}

\noindent Again, Holmes's answers show that the question \textit{Did Moriarty's men terrify everyone?} has the CLARIFICATION, SPEAKER and DIFFERENT ANSWERS properties. And again, what the quantifier domain would be if Holmes were to use the word `everyone', namely all the royal family, is irrelevant to how he should answer a question involving `everyone'. That's the distinctive feature of expressions for which Type B pragmatism is true, and it suggests that in this respect at least `everyone' behaves as if Type B pragmatism is true of it.


\section{Questions about Knowledge}

We have two reasons for thinking that if Type B pragmatism is true about `knows', then questions about knowledge should have all the SPEAKER, CLARIFICATION and DIFFERENT ANSWERS properties. First, the assumption that correct answers are true answers plus trivial facts about the environment (namely that environments are not always fully known and differ between speakers) implies that the questions have these properties. Second, many words that are either uncontroversial or controversial instances where Type B pragmatism is true generate questions with these properties. So if `knows' is meant to be analogous to these controversial examples, questions about knowledge should have these properties. I'll argue in this section that knowledge questions do not have these properties. Again we'll work through a long example to show this.

Last week Watson discovered where Moriarty was storing a large amount of gold, and retrieved it. Moriarty is now coming to Baker St to try to get the gold back, and Holmes is planning a trap for him. Moriarty has made educated guesses that it was Watson (rather than Holmes) who retrieved the gold, and that Holmes is planning a trap at Baker St Station. But he doesn't have a lot of evidence for either proposition. Holmes has been spying on Moriarty, so he knows that Moriarty is in just this position. Neither Moriarty nor Holmes care much about who it was who retrieved the gold from Moriarty's vault, but this is very important to Watson, who plans to write a book about it. On the other hand, that Holmes is planning a trap at Baker St Station is very important to both Holmes and Moriarty, but surprisingly unimportant to Watson. He would prefer that he was the hero of the week for recovering the gold, not Holmes for capturing Moriarty. They have this conversation.

\begin{quote}
Watson: Does Moriarty know that you've got a trap set up at Baker St Station?

Holmes: No, he's just guessing. If I set up a diversion I'm sure I can get him to change his mind.

Watson: Does he know it was me who recovered the gold?

Holmes: Yes, dear Watson, he figured that out.
\end{quote}

These answers sound to me like the answers Holmes \textit{should} give. Because the trap is practically important to both him and Moriarty, it seems he should say no to the first question unless Moriarty has very strong evidence.\footnote{The Type A pragmatist says that it is the importance of the trap to Moriarty that makes this a high-stakes question, and the Type C pragmatist (at least as I understand that view) says that it is the importance of the trap to Holmes that matters. Since Moriarty and Holmes agree about what is important here, the Type A and Type C pragmatists can agree with each other and disagree with the Type B pragmatist.} But because it is unimportant to Holmes and Moriarty just what \textit{Watson} did, the fact that Moriarty has a true belief that's based on the right evidence that Watson recovered the gold is sufficient for Holmes to answer the second question ``Yes''. This shows that questions involving `knows' do not have the SPEAKER property.

Some might dispute the intuitions involved here, but note that on a simple Type B pragmatist theory, one that sets the context just by the \textit{speaker's} interests (as opposed to the speaker \textit{and} her interlocutors) Holmes should assent to (4) and (5).

\numbex{3}{
\item Moriarty does not know that I've got a trap set up for him at Baker St Station, he's just guessing.
\item Moriarty does know that Watson recovered the gold.
}

\noindent And it is \textit{very} intuitive that if he should assent to (4) and (5), then he should answer ``No'' to Watson's first question, and ``Yes'' to the second.

Some adherents of a more sophisticated Type B pragmatist theory will not say that Holmes should assent to these two, because they think he should take Watson's interests on board in his use of `knows'. This is the `single scoreboard view' of Keith \citet{DeRose2004}.\footnote{The failure of NEGATIVE AGREEMENT for knowledge questions suggests that if any contextualist theory is true, it had better be a single scoreboard view. } But we can, with a small addition of complexity rearrange the case so as to avoid this complication. Imagine that Watson is not talking to Holmes, but to Lestrade, and that Lestrade (surprisingly) shares Watson's interests. Unbeknownst to the two of them, Holmes is listening in to their conversation via a bug. When listening in to conversations, Holmes has the habit of answering any questions that are asked, even if they obviously aren't addressed to him. (I do the same thing when watching sports broadcasts, though the questions are often mind-numbingly bland. Listening to the intuitive answers one gives is a good guide to the content of the question.) So the conversation now goes as follows.

\begin{quote}
Watson: Does Moriarty know that Holmes has got a trap set up at Baker St Station?

Holmes (eavesdropping): No, he's just guessing. If I set up a diversion I'm sure I can get him to change his mind.

Lestrade: I'm not sure. My Moriarty spies aren't doing that well.

Watson: Does he know it was me who recovered the gold?

Holmes (eavesdropping): Yes, dear Watson, he figured that out.
\end{quote}

It is important here that Holmes is talking \textit{to himself}, even though he is using Watson's questions to guide what he says. So it will be a \textit{very} extended sense of context if somehow Watson's interests guide Holmes's context. Some speakers may be moved by empathy to align their interests with the people they are speaking about, but \textit{Holmes} is not such a speaker. So Type B pragmatic theories should say that Holmes should endorse (4) and (5) in this context, and intuitively if this is true he should speak in the above interaction just as I have recorded. But this is just to say that questions about knowledge do not have SPEAKER.\footnote{As noted above, the Type A and Type C pragmatists agree with this intuition, though for very different reasons.}

One might worry that we're changing the rules, since we did not use eavesdropping situations above in arguing that if Type B pragmatism is true of `knows', then questions about knowledge should have SPEAKER. But it is a simple exercise to check that even if Holmes is eavesdropping, his answers to questions including `tall', `empty', `ready' and `everyone' should still have the three properties. So this change of scenery does not tilt the deck against the Type B pragmatist.

Cases like this one also suggest that questions involving `knows' also lack the CLARIFICATION and DIFFERENT ANSWERS property. It would be odd of Holmes to reply to one of Watson's questions with ``How much does it matter to you?'', as it would be in any case where a questioner who knows that \textit{p} asks whether \textit{S} also knows that \textit{p}. Intuitions about cases where the questioner does not know that \textit{p} are a little trickier, because then the respondent can only answer `yes' if she has sufficient evidence to assure the speaker that \textit{p}, and it is quite plausible that the amount of evidence needed to assure someone that \textit{p} varies with the interests of the person being assured.\footnote{With the right supplementary assumptions, I believe this claim can be argued to be a consequence of any of our three types of pragmatism.} But if Type B pragmatism were true, we'd expect to find cases of CLARIFICATION where it is common ground that the speaker knows \textit{p}, and yet a request for standards is in order, and no such cases seem to exist. Similarly, it's hard to imagine circumstances where Holmes would offer a different answer if Lestrade rather than Watson asked the questions. And more generally, if two questioners who each know that \textit{p} ask whether S knows that \textit{p}, it is hard to see how it could be apt to answer them differently.\footnote{The above remarks about cases where the speakers don't know that \textit{p} also apply. I think that without a very careful story about how the norms governing assertion relate to the interests of the speaker and the audience, such cases will not tell us much about the semantics of `knows'. Best then to stick with cases where it is common ground that everyone, except the subject, knows that \textit{p}.} But if Type B pragmatism about `knows' were true, knowledge questions would have these three properties, so it follows that Type B pragmatism about `knows' is false.

\section{Objections and Replies}

\noindent \textit{Objection}: The argument that Type B pragmatism implies that questions involving `knows' should have the three properties assumes that correct answers are true answers. But there are good Gricean reasons to think that there are other standards for correctness.

\medskip \noindent \textit{Reply}: It is true that one of the arguments for thinking that Type B pragmatism has this implication uses this assumption. But the other argument, the argument from analogy with indexicals and other terms for which Type B pragmatism is true, does not. Whatever one thinks of the theory, there is data here showing that `knows' does not behave in questions like other context-sensitive terms for which Type B pragmatism is the most plausible view.

Moreover, Type B pragmatists have to be very careful wielding this objection. If there is a substantial gap between correct answers to knowledge questions and true answers, then it is likely that there is a substantial gap between correct knowledge ascriptions and true knowledge ascriptions. As a general (though not universal) rule, truth and correctness are more tightly connected for questions than for simple statements. For example, some utterances do not generate all the scalar implicatures as answers to questions that they typically generate when asserted unprompted.\footnote{Here are a couple of illustrations of this point. Yao Ming is seven feet six inches tall. It would be odd (at best) to use (6) to describe him, but it is clearly improper to answer (7) with `No'.\numbex{5}{\item Yao Ming is over six feet tall. \item Is Yao Ming over six feet tall?} \par \noindent For a second case, consider an example of Hart's that \citet{Grice1989} uses to motivate his distinction between semantics and pragmatics. A motorist drives slowly down the street, pausing at every driveway to see if anyone or anything is rushing out. It seems extremely odd to use (8) to describe him. \numbex{7}{\item He drove carefully down the street.}\par \noindent Is this oddity due to (8) being false, or it being otherwise infelicitous? Part of the argument that (8) is true, but infelicitous, is that intuitively it is correct to say `Yes' in response to (9), and incorrect to say `No'.\numbex{8}{\item Did he drive carefully down the street?}\par \noindent The implicatures associated with (8) are largely absent from affirmative answers to (9), though such an answer presumably has the same truth-conditional content as (8).} But the primary `ordinary language' argument for Type B pragmatism about knowledge ascriptions assumes that correct knowledge ascriptions are, by and large, true.

We can put this in more theoretical terms. If we are to use these considerations to generate a \textit{rebutting} defeater for Type B pragmatism, we would need an argument that Holmes's answers are correct iff they are true. And while that step of the argument is plausible, it is not beyond contention. But that isn't the only use of the examples. We can also use them to \textit{undercut} the argument from ordinary language to Type B pragmatism. We have a wide range of cases where ordinary usage is as if Type B pragmatism is false. And these cases are not peripheral or obscure features of ordinary usage. Answering questions is one of the most common things we do with language. So ordinary usage doesn't provide an all things considered reason to believe that Type B pragmatism is true. If ordinary usage is (or at least was) the best reason to believe Type B pragmatism, it follows that there is no good reason to believe Type B pragmatism.

\bigskip \noindent \textit{Objection}: Sometimes when we ask \textit{Does S know that p?} all we want to know is whether \textit{S} has the information that \textit{p}. In this mood, questions of justification are not relevant. But Type B pragmatism is a theory about the interaction between the subject's \textit{justification} and knowledge ascriptions. So these questions are irrelevant to evaluating indexicalism.\footnote{A similar view is defended by Alvin \citet{Goldman2002}. }

\medskip \noindent \textit{Reply}: It is plausible that there is this use of knowledge questions. It seems to me that this is a usage that needs to be explained, and isn't easily explained on current theories of knowledge.\footnote{One explanation, due to Stephen \citet{Hetherington2001}, is that all that is ever required for a \textit{true} ascription of knowledge that \textit{p} to S is that \textit{p} is true and S believe it. This leaves open the question, as discussed below, of why we sometimes deny that true believers are knowers, but we can bring out familiar Gricean explanations about why we might usefully deny something that is strictly speaking true, or we could follow Hetherington and offer an explanation in terms of the gradability of knowledge claims. Another possible explanation is that `know' univocally means something like what most epistemologists say that it means, but there is a good pragmatic story about why speakers sometimes attribute knowledge to those with merely true belief. (I don't know how such an explanation would go, especially if Type B pragmatism is not true.) Finally, we might follow Goldman and say the English word `know' is ambiguous between a weak and strong reading, and the strong reading has been what has concerned epistemologists, and the weak reading just requires true belief. Such an ambiguity would be a small concession to Type B pragmatism (since it is the interests of the speaker that resolve ambiguities) but we could still sensibly ask whether Type B pragmatism is true of the strong reading.} But I'll leave discussion of that use of knowledge questions for another day. For now I'll just note that even if this can be an explanation of why we sometimes \textit{assent} to knowledge questions, it can't be an explanation of why Holmes \textit{denies} that Moriarty knows about the planned trap at Baker St Station. Holmes agrees that Moriarty \textit{believes} there is a trap planned, but insists that because Moriarty is `just guessing' that this belief does not amount to knowledge. What really needs explaining is the difference between Holmes's two answers, and this other use of knowledge questions doesn't seem sufficient to generate that explanation.

\bigskip \noindent \textit{Objection}: There is no explanation offered here for the data, and we shouldn't give up an explanatory theory without an explanation of why it fails.

\medskip \noindent \textit{Reply}: The best way to explain the data is to look at some differences in our attitudes towards questions containing `knows' and questions containing other terms for which Type B pragmatism is plausibly true. I'll just go over the difference between `knows' and `ready', the point I'm making easily generalises to the other cases. 

Different people may be concerned with different bits of preparation, so they may (speaker) mean different things by \textit{X is ready}. But neither will regard the other as making a \textit{mistake} when they focus on a particular bit of preparing to talk about by saying \textit{X is ready}. And this is true even if they think that the person \textit{should} be thinking about something else that X is preparing. 

Knowledge cases are not like that. Different people may have different standards for knowledge, so perhaps they may (speaker) mean different things by \textit{A knows that p}, because they will communicate that \textit{A} has met their preferred standards for knowledge. But in these cases, each will regard the other as making a \textit{mistake}. Standards for knowledge aren't the kind of thing we say people can differ on without making a mistake, in the way we do (within reason) say that different people can have different immediate goals (e.g. about what to have for dinner) without making a mistake. That explains why we don't just adopt our questioners standards for knowledge when answering their knowledge questions.

\bigskip \noindent \textit{Objection}: In section 2 it was argued that questions involving `knows' should have the three properties because questions involving other terms for which Type B pragmatism is true have the properties. But this argument by analogy may be flawed. All those terms are `indexical' in John MacFarlane's sense \citep{MacFarlane2009-MACNC}. That is, the \textit{content} of any utterance of them varies with the context. But not all Type B pragmatist theories are indexical, and this argument does not tell against a non-indexical Type B pragmatism.

\medskip \noindent \textit{Reply}: The view under consideration says that all utterances of `\textit{S} knows that \textit{p}' express the same proposition, namely that \textit{S} knows that \textit{p}. But the view is Type B because it says that proposition can be true relative to some contexts and false relative to other contexts, just as temporalists about propositions say that a proposition can be true at some times and false at other times, and the utterance is true iff the proposition is true in the context of the utterance. This \textit{is} a Type B view, and I agree that the argument by analogy in section 2 is powerless against it.

But that argument was not the only argument that Type B views are committed to questions involving `knows' having the three properties. There was also an argument, at the end of section 1, from purely theoretical considerations. It turns out this argument is also complicated to apply here, but it does tell against this type of Type B pragmatist as well.

Here are two hypotheses about how one should answer a question \textit{Does NP VP?} where NP is a noun phrase and VP a verb phrase. Consider the proposition expressed by the sentence \textit{NP VP} in the speaker's context. First hypothesis: the correct answer is `Yes' iff that proposition is true in the speaker's context. Second hypothesis: the correct answer is `Yes' iff that proposition is true in the respondent's context. If propositions do not change their truth value between contexts, these two hypotheses are equivalent, but on this version of Type B pragmatism that is not so, so we have to decide between them. The best way to do this is by thinking about cases where it is common ground that propositions change their truth value between contexts, namely cases where the contexts are possible worlds. Let's consider the scenario I briefly discussed above of the television watcher answering whatever question gets asked. Above I put the questioners in the same possible world as me, but we can imagine they are fictional characters in a different possible world. Imagine a show where it has just been revealed to the audience, but not all the characters, that the Prime Minister is a space alien. The following happens.

\begin{quote}
Character (on screen): Is the Prime Minister a space alien?

Me (in real world): Yes!
\end{quote}

\noindent Since the proposition that the Prime Minister is a space alien is true in their world, but not in our world, the propriety of this response tells in favour of the first hypothesis above. Now this is not a conclusive argument, because it might be that there is some distinctive feature of fiction that is causing this answer, even though the second hypothesis is in general correct. But it seems we have reason to think that \textit{if} this kind of Type B pragmatism were true, respondents would use the speaker's context to work out what kind of answer to give. And as we saw above, this is not what respondents \textit{actually} do, they either use their own context or (more likely) the subject's.

\bigskip \noindent \textit{Objection}: At most this shows that Type B pragmatism about `knows' is not actually true of English. But there might still be good epistemological reasons to adopt it as a philosophically motivated revision, even if the data from questions shows it isn't true. So even if hermeneutic Type B pragmatism is false, revolutionary Type B pragmatism might be well motivated.

\medskip \noindent \textit{Reply}: It's true that the philosophically most interesting concepts may not map exactly on to the meanings of words in natural language. (Though I think we should be careful before abandoning the concepts that have proven useful enough to get simple representation in the language.) And it's true that there are reasons for having epistemological concepts that are sensitive to pragmatic factors. But what is hard to see is what interest we could have in having epistemological terms whose application is sensitive to the interests of the person using them. \citet{Hawthorne2004} provides many reasons for thinking that terms whose applications are sensitive to the interests of the person to whom they are being applied are philosophically and epistemologically valuable. Such terms provide ways of expressing unified judgements about the person's intellectual and practical reasoning. On the other hand, there is little to be gained by \textit{adopting} Type B pragmatism. If there needs to be a revision around here, and my guess is that there does not, it should be towards Type A, not Type B, pragmatism.
