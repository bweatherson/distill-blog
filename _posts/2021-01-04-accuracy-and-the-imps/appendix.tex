%\documentclass{article}
%\usepackage{amsmath}
%\usepackage{mdwlist}
%\usepackage{mathrsfs}
%\usepackage{amssymb}
%
%\begin{document}

\newcommand{\ma}{\mathscr{A}}
\newcommand{\sa}{\mathscr{S}}
\newcommand{\ba}{\mathscr{B}}


\section*{Appendix: Proofs of Theorems 1, 2, 3}
\begin{quote}
\textbf{Theorem-1}: Brier$_{\ma}(\mathbf{c},@) = \frac{N}{4}\text{Brier}_{\sa}(\mathbf{c},@)$ where
\begin{align*}
\text{Brier}_{\sa}(\mathbf{c},@) &= \frac{\sum_{s \in \sa} (@(s) - c(s))^2}{N}
\end{align*}
\end{quote}
To prove this we rely on a series of lemmas.\footnote{[Redacted] gives a more direct and elegant proof of this result in a recent manuscript [reference removed for anonymous review]. We have kept our inefficient proof since its structure provides a guide for the proof of Theorem-3.}

Let $\ma$ be the algebra generated by a finite partition of states $\sa = \{s_1, s_2, \dots, s_N\}$. @ is a truth-value assignment for propositions in $\ma$. For simplicity, assume $s_1$ is the true state, so that @($s_1$) = 1 and @($s_n$) = 0 for $n > 1$. The credence function \textbf{c} assigns values of $c_1, c_2, \dots, c_{N-1}, c_N$  to the elements of $\sa$, where $\sum^{N}_{n=1} c_n = 1$ in virtue of coherence.

It will be convenient to start by partitioning $\ma$ into four ''quadrants''. Let $B$ range over all disjunctions with disjunctions drawn from $\ba = \{s_2, s_3, \dots, s_{N-1}\}$ (including the empty disjunction, i.e., the logical contradition $\bot$). Then, $\ma$ can be split into four disjoint parts:

\begin{itemize*}
\renewcommand\labelitemi{}
\item $\ma_1 = \{B \vee s_1 \vee s_N: B$ a disjunction of the elements of $\ba \}$
\item $\ma_2 = \{B \vee s_1: B$ a disjunction of the elements of $\ba \}$
\item $\ma_3 = \{B \vee s_N: B$ a disjunction of the elements of $\ba \}$
\item $\ma_4 = \{B: B$ a disjunction of the elements of $\ba \}$
\end{itemize*}
Notice that:

\begin{enumerate*}
\renewcommand\labelenumi{(\roman{enumi})}
\item $\ma_1 \cup \ma_2$ contains all and only the true propositions in $\ma$.
\item $\ma_3 \cup \ma_4$ contains all and only the false propositions in $\ma$.
\item $\ma_1$ and $\ma_4$ are \emph{complementary} sets, i.e., all elements of $\ma_4$ are negations of elements of $\ma_1$, and conversely. 
\item $\ma_2$ and $\ma_3$ are also complementary.
\item $\ma_1 \cup \ma_4$ is the subalgebra of $\ma$ generated by $\{s_1 \vee s_N, s_2, s_3, \dots, s_{N-1}\}$.
\item All four quadrants have the same cardinality of $2^{N-2}$.
\end{enumerate*}
For an additive scoring rule 
$\mathbf{I}(\mathbf{c}, @) = \sum_{A \in \ma}\mathbf{i}(\mathbf{c}(A), @(A))$ 
and $j = 1, 2, 3, 4$, define 
$\mathbf{I}_j = \sum_{A \in \ma_j}\mathbf{i}(\mathbf{c}(A), @(A))$, 
and note that $\mathbf{I}(\mathbf{c}, @) = 2^{-N}(\mathbf{I}_1 + \mathbf{I}_2 + \mathbf{I}_3 + \mathbf{I}_4)$.

\begin{quote}
\textbf{Lemma-1.1}: If $\textbf{I}$ is negation symmetric, i.e., if $\mathbf{i}(\mathbf{c}(\neg A), @(\neg A)) = \mathbf{i}(\mathbf{c}(A), @(A))$ for all $A$, then $\mathbf{I}_1 = \mathbf{I}_4$ and $\mathbf{I}_2 = \mathbf{I}_3$, and $\mathbf{I}(\mathbf{c},@) = 2^{1-N}(\mathbf{I}_2 + \mathbf{I}_4)$. 
\end{quote}
\textbf{Proof}: This is a direct consequence of the fact that $\ma_1$ is complementary to $\ma_4$ and that $\ma_2$ is complementary to $\ma_3$ since this allows us to write

\begin{align*}
\mathbf{I}_1(\mathbf{c},@) &= \sum_{A \in \ma_1} \mathbf{i}(\mathbf{c}(A), @(A)) = \sum_{A \in \ma_1} \mathbf{i}(\mathbf{c}(\neg A), @(\neg A)) = \mathbf{I}_4(\mathbf{c},@). \\
\mathbf{I}_3(\mathbf{c},@) &= \sum_{A \in \ma_3} \mathbf{i}(\mathbf{c}(A), @(A)) = \sum_{A \in \ma_3} \mathbf{i}(\mathbf{c}(\neg A), @(\neg A)) = \mathbf{I}_2(\mathbf{c},@). \text{ QED}\\
\end{align*}
Applying Lemma 1.1 with \textbf{I} = \textbf{Brier} we get

\begin{align*}
(\#)\quad \mathbf{Brier}_{\ma}(\mathbf{c}, @) &= 2^{1-N} \sum_{A \in \ma} (@(A) - c(A))^2 \\
 &= 2^{1-N} \sum_B [(1-c_1)^2 - 2(1-c_1)\mathbf{c}(B) + \mathbf{c}(B)^2]
\end{align*}
since

\begin{alignat*}{3}
\mathbf{Brier}_2 &= \sum_B[1 - \mathbf{c}(B \vee s_1)]^2 &&= \sum_B[(1 - c_1) - \mathbf{c}(B)]^2 \\
& &&= \sum_B [(1-c_1)^2 - 2(1-c_1)\mathbf{c}(B) + \mathbf{c}(B)^2] \\
\mathbf{Brier}_4 &= \sum_B \mathbf{c}(B)^2 && \quad
\end{alignat*}

\begin{quote}
\textbf{Lemma-1.2} 
\begin{align*}
(\sum_{n=2}^{N-1} c_n)^2 &= \sum_{n=2}^{N-1} c{_n}^2 + 2 \sum_{n=2}^{N-2} \sum_{j>n}^{N-1}c_nc_j
\end{align*}
\end{quote}
Proof by induction. Easy. 

\begin{quote}
\textbf{Lemma-1.3}
\begin{align*}
\mathbf{Brier}_{\sa}(\mathbf{c},@) &= \frac{2}{N}[(1-c_1)^2 + \sum_{n=2}^{N-1} c{_n}^2  - (1-c_1)(\sum_{n=2}^{N-1}c_n) + \sum_{n=2}^{N-2} \sum_{j>n}^{N-1}c_nc_j]
\end{align*}
\end{quote}
\textbf{Proof}: Using the definition of the Brier score and the fact that $s_1$ is true, we have
 
\begin{align*}
\mathbf{Brier}_{\sa}(\mathbf{c},@) &= \frac{1}{N}[(1 - c_1)^2 + \sum_{n=2}^{N-1} c{_n}^2 + (1 - \sum_{n=1}^{N-1}c_n)^2] \\
&= \frac{1}{N}[(1 - c_1)^2 + \sum_{n=2}^{N-1} c{_n}^2 + ((1 -c_1) - \sum_{n=2}^{N-1}c_n)^2] \\
&= \frac{1}{N}[(1 - c_1)^2 + \sum_{n=2}^{N-1} c{_n}^2 + (1 -c_1)^2 - 2(1 - c_1) \sum_{n=2}^{N-1}c_n + (\sum_{n=2}^{N-1}c_n)^2] \\
&= \frac{1}{N}[(1 - c_1)^2 + \sum_{n=2}^{N-1} c{_n}^2 + (1 -c_1)^2 - 2(1 - c_1) \sum_{n=2}^{N-1}c_n  \\
&\quad \quad +\sum_{n=2}^{N-1} c{_n}^2 + 2 \sum_{n=2}^{N-2} \sum_{j>n}^{N-1}c_nc_j] \quad \text{(Lemma-1.2)}
\end{align*}
Then grouping like terms and factoring out 2 yields the desired result. QED

\begin{quote}
\textbf{Lemma-1.4}
\begin{align*}
\sum_{n=2}^{N-1}c_n &= 2^{3-N}\sum_{B \in \ba}\mathbf{c}(B)
\end{align*}
\end{quote}
\textbf{Proof}: For each $n = 2, 3, \dots, N-1$, each $s_n$ appears in half of the $2^{N-2}$ disjunctions with disjuncts drawn from $\ba$. As a result, each $c_n$ appears as a summand $2^{N-3}$ times among the sums that express the various $\mathbf{c}(B)$. So $\sum_{B \in \ba}\mathbf{c}(B) = 2^{N-3}\sum_{n=2}^{N-1}c_n$. QED

\begin{quote}
\textbf{Lemma-1.5}
\begin{align*}
\sum_{B \in \ba}\mathbf{c}(B)^2 &= 2^{N-3}[\sum_{n=2}^{N-1} c{_n}^2 + \sum_{n=2}^{N-2} \sum_{j>n}^{N-1}c_nc_j]
\end{align*}
\end{quote}
\textbf{Proof}: We proceed by induction starting with the first meaningful case of $N=4$, where calculation shows $\sum_B\mathbf{c}(B)^2 = (c_2 + c_3)^2 + c{_2}^2 + c{_3}^2 = 2[c{_2}^2 + c{_3}^2 + c_2c_3]$. Now, assume the identity holds for disjunctions $B$ of elements of $\ba$ and show that it holds for disjunctions $A$ of elements of $\ba \cup \{s_N\}$.

\begin{align*}
\sum_A\mathbf{c}(A)^2 &= \sum_B\mathbf{c}(B)^2 + \sum_B\mathbf{c}(B \vee s_N)^2 \\
&= \sum_B\mathbf{c}(B)^2 + \sum_B(\mathbf{c}(B)^2 + 2c_N\mathbf{c}(B) + c{_N}^2)\\
&= 2\sum_B\mathbf{c}(B)^2 + 2c_N\sum_B\mathbf{c}(B) + \sum_Bc{_N}^2 \\
&= 2 \cdot 2^{N-3}[\sum_{n=2}^{N-1} c{_n}^2 + \sum_{n=2}^{N-2} \sum_{j>n}^{N-1}c_nc_j] + 2c_N\sum_B\mathbf{c}(B) + \sum_Bc{_N}^2 \\
\shortintertext{\flushright (Induction Hypothesis)} \\
&= 2^{N-2}[\sum_{n=2}^{N-1} c{_n}^2 + \sum_{n=2}^{N-2} \sum_{j>n}^{N-1}c_nc_j] + 2^{N-2}c_N\sum_{n=2}^{N-1}c_n + \sum_Bc{_N}^2 \\
\shortintertext{\flushright (Lemma-1.4)} \\
&= 2^{N-2}[\sum_{n=2}^{N-1} c{_n}^2 + \sum_{n=2}^{N-2} \sum_{j>n}^{N-1}c_nc_j] + 2^{N-2}c_N\sum_{n=2}^{N-1}c_n + 2^{N-2}c{_N}^2 \\
\shortintertext{\flushright Since $|\ba| = 2^{N-2}$} \\
&= 2^{N-2}[\sum_{n=2}^{N} c{_n}^2 + \sum_{n=2}^{N-1} \sum_{j>n}^{N}c_nc_j] && \text{ QED}
\end{align*}
Plugging the results of the last two lemmas into Lemma-1.3 produces a result of

\begin{align*}
\mathbf{Brier}_{\sa}(\textbf{c},@) &= \frac{2}{N}[(1-c_1)^2 + 2^{3-N}\sum_{B \in \ba}\mathbf{c}(B)^2 - 2^{3-N}(1-c_1)\sum_{B \in \ba}\mathbf{c}(B)] \\
&= \frac{2}{N}\sum_{B \in \ba}[2^{2-N}(1-c_1)^2 + 2^{3-N}\mathbf{c}(B)^2 - 2^{3-N}(1-c_1)\mathbf{c}(B)] \\
&= \frac{2^{3-N}}{N}\sum_{B \in \ba}[(1 - c_1)^2 + 2\mathbf{c}(B)^2 - 2(1-c_1)\mathbf{c}(B)]
\end{align*}
Comparing this to (\#) we see that it is just $\frac{N}{4}$ times Brier$_\sa(\mathbf{c},@)$, as we aimed to prove. QED.

\begin{quote}
\textbf{Theorem-2}. When inaccuracy over $\ma$ is measured using the Brier score, the least accurate credal states are those which assign credence 1 to some false atom of $\ma$.
\end{quote}
\textbf{Proof}: As before, suppose that $@(s_1) = 1$, and let \textbf{c} be a credence function that assigns credence 1 to some false atom $s_2, s_3,..., s_N$ of $\ma$. In light of Theorem-1 it suffices to show that $\mathbf{Brier}_\sa(\mathbf{c}, @) > \mathbf{Brier}_\sa(\mathbf{b}, @)$ where \textbf{b} does not assign credence 1 to any false atom. Start by noting that for any credence function $\pi$ defined on the atoms of $\ma$ one has

\begin{align*}
\mathbf{Brier}_\sa(\pi,@) &= \frac{1}{N}[(1-\pi_1)^2 + \sum_{n=2}^{N-1}\pi{_n}^2+ (1-\sum_{n=1}^{N-1}\pi{_n})^2] \\
&= \frac{1}{N}[1 - 2\pi_1 + \sum_{n=1}^{N-1}\pi{_n}^2+ (1-\sum_{n=1}^{N-1}\pi{_n})^2]
\end{align*}

But, since each $\pi_n \in [0, 1]$ is non-negative, it follows that $\pi_1 \geq \pi{_1}^2, \pi_2 \geq \pi{_2}^2, \dots, \pi_N \geq \pi{_N}^2$ with the inequality strict in each case unless $\pi_n$ is either 1 or 0.

This means that the sum $\sum_{n=1}^{N-1}\pi{_n}^2+ (1-\sum_{n=1}^{N-1}\pi{_n})^2$ is less than or equal to 1, with equality if and only if exactly one of the atoms $s_n$ is assigned probability 1 (and the rest have probability zero). As a result, \textbf{Brier}$_\sa(\pi, @) \leq \frac{2}{N}(1 - \pi_1)$ with equality if and only if exactly one of the atoms $s_n$ is assigned probability 1. So, there are three relevant cases:

\begin{enumerate*}
\renewcommand{\labelenumi}{(\roman{enumi})}
\item If $\pi$ assigns some false atom probability 1, \textbf{Brier}$_\sa(\pi, @) = \frac{2}{N}\cdot(1 - 0) =  \frac{2}{N}$.
\item If $\pi$ assigns the true atom probability 1, \textbf{Brier}$_\sa(\pi, @) = \frac{2}{N}\cdot(1 - 1) = 0$.
\item If $\pi$ does not assign any atom probability 1, \textbf{Brier}$_\sa(\pi, @) < \frac{2}{N}\cdot(1 - c_1) \leq \frac{2}{N}$.
\end{enumerate*}
So, since \textbf{c} fits case (i) and \textbf{b} fits case (ii) or (iii) we have the desired result. QED

\begin{quote}
\textbf{Theorem-3}: Let $\ma$ be an algebra of propositions generated by atoms $a_1, ..., a_{2N}$, where $a_1$ is the truth.  Let $P$ and $Q$ be probability functions defined on $\ma$.  $P$ assigns all its mass to the first $N$ atoms, so that $P(a_1 \vee \dots \vee a_N) = 1$, and it also assigns some positive probability to $a_1$.  $Q$ assigns all its mass to the false atom $a_{2N}$, so that $Q(a_{2N}) = 1$.  Then, for any proper score \textbf{I} satisfying Truth-directedness, Extensionality and Negation Symmetry we have $\mathbf{I}(v_1, P) < \mathbf{I}(v_1, Q)$ where $v_1$ is the truth-value assignment associated with $a_1$ (i.e., where $v_1(X) = 1$ if and only if $a_1$ entails $X$).
\end{quote}
\textbf{Proof}: We can divide the algebra $\mathscr{A}$ into four quadrants
\begin{align*}
\mathscr{A}^1 &= \{X \in \mathscr{A}: a_1 \vDash X \text{ and } a_{2N} \vDash X\} \\
\mathscr{A}^2 &= \{X \in \mathscr{A}: a_1 \vDash X \text{ and } a_{2N} \nvDash X\} \\
\mathscr{A}^3 &= \{X \in \mathscr{A}: a_1 \nvDash X \text{ and } a_{2N} \vDash X\} \\
\mathscr{A}^4 &= \{X \in \mathscr{A}: a_1 \nvDash X \text{ and } a_{2N} \nvDash X\}
\end{align*}
We know the following:

\begin{itemize}
\item $Q$ is maximally accurate on $\mathscr{A}^1 \cup \mathscr{A}^4$. Every proposition in $\mathscr{A}^1$ is true, and $Q$ assigns it a probability of 1. Every proposition in $\mathscr{A}^4$ is false, and $Q$ assigns it a probability of 0.

\item $Q$ is maximally inaccurate on $\mathscr{A}^2 \cup \mathscr{A}^3$. Every proposition in $\mathscr{A}^2$ is true, and $Q$ assigns it a probability of 0. Every proposition in $\mathscr{A}^3$ is false, and $Q$ assigns it a probability of 1.

\item $P$ is maximally accurate on $\mathscr{A}^3 \cup \mathscr{A}^4$. Every proposition in $\mathscr{A}^3 \cup \mathscr{A}^4$ is false, and $P$ assigns it a probability of 0.

\item Each quadrant has $2^{2N-2}$ elements.

\end{itemize}
\begin{quote}

\textbf{Lemma-3.1}: When $a_1$ is true, the accuracy score of $P$ over the propositions in $\mathscr{A}^1$ is identical to the accuracy score of $P$ over the propositions in $\mathscr{A}^2$.
\end{quote}
\textbf{Proof}: Note first that the function $F: \mathscr{A}^1 \rightarrow \mathscr{A}^2$ that takes $X$ to $X \wedge \neg a_{2N}$ is a bijection of $\mathscr{A}^1$ onto $\mathscr{A}^2$. Since every proposition in $\mathscr{A}^1 \cup \mathscr{A}^2$ is true, we can then write the respective accuracy scores of $\mathscr{A}^1$ and  $\mathscr{A}^2$ as 
\begin{align*}
\mathbf{I}_{\mathscr{A}^1}(a_1, P) &= 2^{2-2N} \cdot \sum_{X \in \mathscr{A}^1} \mathbf{I}(1, P(X)) \\
\mathbf{I}_{\mathscr{A}^2}(a_1, P) &= 2^{2-2N} \cdot \sum_{X \in \mathscr{A}^1} \mathbf{I}(1, P(X \wedge \neg a_{2N}))
\end{align*}
Note: $X$ ranges over $\mathscr{A}^1$ in both summations. But since $P(a_{2N}) = 0$ we have $P(X) = P(X \wedge a_{2N})$ for each $X$ in $\mathscr{A}^1$. Since \textbf{I} is extensional, this means that $\mathbf{I}(1, P(X)) = \mathbf{I}(1, P(X \wedge a_{2N}))$ for each $X$ in $\mathscr{A}^1$. And, it follows that $\mathbf{I}_{\mathscr{A}^1}(a_1, P)$ and $\mathbf{I}_{\mathscr{A}^2}(a_1, P)$ are identical. (Note that even if $P(a_{2N}) > 0$, Truth-directedness entails that $\mathbf{I}_{\mathscr{A}^1}(a_1, P) < \mathbf{I}_{\mathscr{A}^2}(a_1, P)$.)

\begin{quote}

\textbf{Lemma-3.2}: When $a_1$ is true, the accuracy score of $Q$ over $\mathscr{A}^2$ is identical to the accuracy score of $Q$ over $\mathscr{A}^3$.
\end{quote}
\textbf{Proof}: To see this, note first that the function $G: \mathscr{A}^2 \rightarrow \mathscr{A}^3$ that takes $X$ to $G(X) = \neg X$ is a bijection (i.e., the negation of everything in $\mathscr{A}^2$ is in $\mathscr{A}^3$ and vice-versa). This, together with the fact that $\mathscr{A}^2$ contains only truths and $\mathscr{A}^3$ contains only falsehoods, lets us write
\begin{align*}
\mathbf{I}_{\mathscr{A}^2}(a_1, Q) &= 2^{2-2N} \cdot \sum_{X \in \mathscr{A}^2} \mathbf{I}(1, Q(X)) \\
\mathbf{I}_{\mathscr{A}^3}(a_1, Q) &= 2^{2-2N} \cdot \sum_{X \in \mathscr{A}^2} \mathbf{I}(0, Q(\neg X))
\end{align*}
But since \textbf{I} is negation symmetric, $\mathbf{I}(1, Q(X)) = \mathbf{I}(0, Q(\neg X))$ for every $X$, which means that $\mathbf{I}_{\mathscr{A}^2}(a_1, Q) = \mathbf{I}_{\mathscr{A}^3}(a_1, Q)$. (Note that this proof made no assumptions about $Q$ except that it was a probability.)

\begin{quote}

\textbf{Lemma-3.3}: If $P(a_1) > 0$, the accuracy score of $P$ over $\mathscr{A}^2$ is strictly less than the accuracy score of $Q$ over $\mathscr{A}^2$.
\end{quote}
\textbf{Proof}: Since $Q(X) = 0$ everywhere on $\mathscr{A}^2$ we have
\begin{align*}
\mathbf{I}_{\mathscr{A}^2}(a_1, P) &= 2^{2-2N} \cdot \sum_{X \in \mathscr{A}^2} \mathbf{I}(1, P(X)) \\
\mathbf{I}_{\mathscr{A}^2}(a_1, Q) &= 2^{2-2N} \cdot \sum_{X \in \mathscr{A}^2} \mathbf{I}(1, 0) 
\end{align*}
But, by Truth Directedness $\mathbf{I}(1, 0) > \mathbf{I}(1, P(X))$ since $P(a_1) > 0$ implies that $P(X) > 0$ for all $X \in \mathscr{A}^2$. Thus $\mathbf{I}_{\mathscr{A}^2}(a_1, Q) > \mathbf{I}_{\mathscr{A}^2}(a_1, P)$.

\bigskip

To complete the proof of the theorem we need only note that
\begin{align*}
\mathbf{I}_{\mathscr{A}}(a_1, P) &= \frac{\mathbf{I}_{\mathscr{A}^1}(a_1, P)}{4} + \frac{\mathbf{I}_{\mathscr{A}^2}(a_1, P)}{4} &\text{(since }P\text{ is perfect on }\mathscr{A}^3 \cup \mathscr{A}^4) \\
&= \frac{\mathbf{I}_{\mathscr{A}^2}(a_1, P)}{2} &\text{Lemma-3.1} \\
&< \frac{\mathbf{I}_{\mathscr{A}^2}(a_1, Q)}{2} &\text{Lemma-3.3} \\
&= \frac{\mathbf{I}_{\mathscr{A}^2}(a_1, Q)}{4} + \frac{\mathbf{I}_{\mathscr{A}^3}(a_1, Q)}{4} &\text{Lemma-3.2} \\
&= \mathbf{I}_{\mathscr{A}}(a_1, Q) &\text{(since }Q\text{ is perfect on }\mathscr{A}^1 \cup \mathscr{A}^4)
\end{align*}

%\end{document}