\chapter{Blome-Tillmann on Interest-Relative Invariantism}
\unpub{}

\noindent Michael Blome-Tillmann has argued that his version of Lewisian contextualism is preferable to interest-relative invariantism \citep{MBT2009}\footnote{Blome-Tillmann calls interest-relative invariantism `subject-sensitive invariantism'. This is an unfortunate moniker. The only subject-\textit{insensitive} theory of knowledge has that for any \(S, T\) $S$ knows that $p$ iff $T$ knows that $p$. The view Blome-Tillmann is targetting, as set out in \citet{Fantl2002, Hawthorne2004, Stanley2005-STAKAP} and \citet{Weatherson2005-WEACWD} certainly isn't defined in opposition to \textit{this} generalisation. The canonical source for Lewisian contextualism is \citet{Lewis1996b}, and Blome-Tillmann defends a variant in \citet{MBT2009a}.}. There are three main arguments offered, one from modal embeddings, one from temporal embeddings and one from deviant conjunctions. I don't think any of these work, for reasons that I'll go through in turn.

\section{Modal Embeddings}
Jason Stanley had argued that the fact that interest-relative invariantism (hereafter, IRI) has counterintuitive consequences when it comes to knowledge ascriptions in modal contexts shouldn't count too heavily against IRI, because contextualist approaches are similarly counterintuitive. In particular, he argues that the theory that `knows' is a contextually sensitive quantifier, plus the account of quantifier-domain restriction that he developed with Zolt\'{a}n Gendler Szab\'{o} \citep{Stanley2000-STAOQD}, has false implications when it is applied to knowledge ascriptions in counterfactuals. Blome-Tillmann disagrees, but I don't think he provides very good reasons for disagreeing. Let's start by reviewing how we got to this point.

Often when we say \textit{All Fs are Gs}, we really mean \textit{All C Fs are Gs}, where $C$ is a contextually specified property. So when I say \textit{Every student passed}, that utterance might express the proposition that every student \textbf{in my class} passed. Now there's a question about what happens when sentences like \textit{All Fs are Gs} are embedded in various contexts. The question arises because quantifier embeddings tend to allow for certain kinds of ambiguity. For instance, when we have a sentence like \textit{If p were true, all Fs would be G}, that could express either of the following two propositions. (We're ignoring context sensitivity for now, but we'll return to it in a second.)

\begin{itemize}
\item If $p$ were true, then everything that would be $F$ would also be $G$.
\item If $p$ were true, then everything that's actually $F$ would be $G$.
\end{itemize}

\noindent We naturally interpret (1) the first way, and (2) the second way.

\numbex{0}{
\item If I had won the last Presidential election, everyone who voted for me would regret it by now.
\item If Hilary Clinton had been the Democratic nominee in the last Presidential election, everyone who voted for Barack Obama would have voted for her.
}

\noindent Given this, you might expect that we could get a similar ambiguity with $C$. That is, when you have a quantifier that's tacitly restricted by $C$, you might expect that you could interpret a sentence like \textit{If p were true, all Fs would be G} in either of these two ways. (In each of these interpretations, I've left $F$ ambiguous; it might denote the actual \(F\)s or the things that would be \(F\) if \(p\) were true. So these are just partial disambiguations.)

\begin{itemize}
\item If $p$ were true, then every $F$ that would be $C$ would also be $G$.
\item If $p$ were true, then every $F$ that is actually $C$ would be $G$.
\end{itemize}

\noindent Surprisingly, it's hard to get the second of these readings. Or, at least, it is hard to \textit{avoid} the availability of the first reading. Typically, if we restrict our attention to the $C$s, then when we embed the quantifier in the consequent of a counterfactual, the restriction is to the things that would be $C$, not to the actual $C$s.\footnote{See \citet{Stanley2000-STAOQD} and \citet{Stanley2005-STAKAP} for arguments to this effect.} 

Blome-Tillmann notes that Stanley makes these observations, and interprets him as moulding them into the following argument against Lewisian contextualism.

\begin{enumerate}
\item An utterance of \textit{If p were true, all Fs would be Gs} is interpreted as meaning \textit{If p were true, then every F that would be C would also be G}.
\item Lewisian contextualism needs an utterance of \textit{If p were true, then S would know that q} to be interpreted as meaning \textit{If p were true, then S's evidence would rule out all $\neg$q possibilities, except those that are actually being properly ignored}, i.e. it needs the contextually supplied restrictor to get its extension from the nature of the actual world.
\item So, Lewisian contextualism is false.
\end{enumerate}

\noindent And Blome-Tillmann argues that the first premise of this argument is false. He thinks that he has examples which undermine premise 1. But I don't think his examples show any such thing. Here are the examples he gives. (I've altered the numbering for consistency with this paper.)

\numbex{2}{
\item If there were no philosophers, then the philosophers doing research in the field of applied ethics would be missed most painfully by the public.
\item If there were no beer, everybody drinking beer on a regular basis would be much healthier.
\item If I suddenly were the only person alive, I would miss the Frege scholars most.
}

\noindent These are all sentences of (more or less) the form \textit{If p were true, Det Fs would be G}, where \(Det\) is some determiner or other, and they should all be interpreted a la our second disambiguation above. That is, they should be interpreted as quantifying over actual $F$s, not things that would be $F$ if $p$ were true. But the existence of such sentences is completely irrelevant to what's at issue in premise 1. The question isn't whether there is an ambiguity in the $F$ position, it is whether there is an ambiguity in the $C$ position. And nothing Blome-Tillmann raises suggests premise 1 is false. So this response doesn't work.

Even if a Lewisian contextualist were to undermine premise 1 of this argument, they wouldn't be out of the woods. That's because premise 1 is much stronger than is needed for the anti-contextualist argument Stanley actually runs. Note first that the Lewisian contextualist needs a reading of \textit{If p were true, all Fs would be G} where it means:

\begin{itemize}
\item If $p$ were true, every actual $C$ that would be $F$ would also be $G$.
\end{itemize}

\noindent The reason the Lewisian contextualist needs this reading is that on their story, \textit{S knows that p} means \textit{Every $\neg p$ possibility is ruled out by S's evidence}, where the \textit{every} has a contextual domain restriction, and the Lewisian focuses on the actual context. The effect in practice is that an utterance of \textit{S knows that p} is true just in case every $\neg p$ possibility that the speaker isn't properly ignoring, i.e., isn't actually properly ignoring, is ruled out by $S$'s evidence. Lewisian contextualism is meant to explain sceptical intuitions, so let's consider a particular sceptical intuition. Imagine a context where:

\begin{itemize}
\item I'm engaged in sceptical doubts;
\item there is beer in the fridge
\item I've forgotten what's in the fridge; and
\item I've got normal vision, so if I check the fridge I'll see what's in it.
\end{itemize}

\noindent In that context it seems (6) is false, since it would only be true if Cartesian doubts weren't salient.

\numbex{6}{
\item If I were to look in the fridge and ignore Cartesian doubts, then I'd know there is beer in the fridge.
}

\noindent But the only way to get that to come out false, and false for the right reasons, is to fix on which worlds we're actually ignoring (i.e., include in the quantifier domain worlds where I'm the victim of an evil demon), but look at worlds that would be ruled out with the counterfactually available evidence. We don't want the sentence to be false because I've actually forgotten what's in the fridge. And we don't want it to be true because I would be ignoring Cartesian possibilities. In the terminology above, we would need \textit{If p were true, all Fs would be Gs} to mean \textit{If p were true, then every actual C that were F would also be G}. We haven't got any reason yet to think that's even a \textit{possible} disambiguation of (6). 

But let's make things easy for the contextualist and assume that it is. Stanley's point is that the contextualist needs even more than this. They need it to be by far the  \textit{preferred} disambiguation, since in the context I describe the natural reading of (6) (given sceptical intuitions) is that it is false because my looking in the fridge wouldn't rule out Cartesian doubts. And they need it to be the preferred reading even though there are alternative readings that are (a) easier to describe, (b) of a kind more commonly found, and (c) true. Every principle of contextual disambiguation we have pushes us away from thinking this is the preferred disambiguation. This is the deeper challenge Stanley raises for contextualists, and it hasn't yet been solved.

\section{Temporal Embeddings}
\noindent Blome-Tillmann thinks that IRI will lead to implausible consequences with temporal embeddings. He illustrates this with a variant of the well-known bank cases. (I assume familiarity with these cases among my readers.) Let $O$ be that the bank is open Saturday morning. If Hannah has a large debt, she is in a high-stakes situation with respect to $O$. She had in fact incurred a large debt, but on Friday morning the creditor waived this debt. Hannah had no way of anticipating this on Thursday. She has some evidence for $O$, but not enough for knowledge if she's in a high-stakes situation. Blome-Tillmann says that this means after Hannah discovers the debt waiver, she could say (7).

\numbex{6}{
\item I didn't know $O$ on Thursday, but on Friday I did.
}

\noindent But I'm not sure why this case should be problematic. As Blome-Tillmann notes, it isn't really a situation where Hannah's stakes change. She was never actually in a high stakes situation. At most her perception of her stakes change; she thought she was in a high-stakes situation, then realised that she wasn't. Blome-Tillmann argues that even this change in perceived stakes can be enough to make (7) true if IRI is true. I agree that this change in perception could be enough to make (7) true, but when we work through the reason that's so, we'll see that it isn't because of anything distinctive, let alone controversial, about IRI.

If Hannah is rational, then given her interests she won't be ignoring $\neg O$ possibilities on Thursday. She'll be taking them into account in her plans. Someone who is anticipating $\neg O$ possibilities, and making plans for them, doesn't know $O$. That's not a distinctive claim of IRI. Any theory should say that if a person is worrying about $\neg O$ possibilities, and planning around them, they don't know $O$. If Hannah is rational, that will describe her on Thursday, but not on Friday. So (7) is true not because Hannah's practical situation changes between Thursday and Friday, but because her psychological state changes.

What if Hannah is, on Thursday, irrationally ignoring $\neg O$ possibilities, and not planning for them even though her rational self wishes she were planning for them? Then Hannah has irrational attitudes towards $O$, and anyone who has irrational attitudes towards $O$ doesn't know $O$, since knowledge requires a greater level of rationality than Hannah shows here. The principle from the last sentence can lead to some quirky results for reasons independent of IRI, and this could trip us up if we spend too much time on particular cases, and too little on epistemological theory. 

So consider Bobby. Bobby has the disposition to infer $\neg B$ from $A \rightarrow B$ and $\neg A$. He currently has good inductive evidence for $q$, and infers $q$ on that basis. But he also knows $p \rightarrow q$ and $\neg p$. If he notices that he has these pieces of knowledge, he'll infer $\neg q$. This inferential disposition defeats any claim he might have to \textit{know} $q$; the inferential disposition is a kind of doxastic defeater. Then Bobby sits down with some truth tables and talks himself out of the disposition to infer $\neg B$ from $A \rightarrow B$ and $\neg A$. He now knows $q$ although he didn't know it earlier, when he had irrational attitudes towards a web of propositions including $q$. And that's true even though his evidence for $q$ didn't change.\footnote{I assume, what shouldn't be controversial, that irrational inferential dispositions which the agent does not know he has, and which he does not apply, are not part of his evidence.} 

Bobby's case is parallel to the case where Hannah irrationally ignores the significance of $O$ to her practical deliberation. In both cases, defective mental states elsewhere in their cognitive architecture defeat knowledge claims. And in that kind of case, we should expect sentences like (7) to be true, even if they appear counterintuitive before we've worked through the details.

\section{Conjunctions}
\noindent George and Ringo both have \$6000 in their bank accounts. They both are thinking about buying a new computer, which would cost \$2000. Both of them also have rent due tomorrow, and they won't get any more money before then. George lives in New York, so his rent is \$5000. Ringo lives in Syracuse, so his rent is \$1000. Clearly, (8) and (9) are true.

\numbex{7}{
\item Ringo has enough money to buy the computer.
\item Ringo can afford the computer.
}

\noindent And (10) is true as well, though there's at least a reading of (11) where it is false.

\numbex{9}{
\item George has enough money to buy the computer.
\item George can afford the computer.
}

\noindent Focus for now on (10). It is a bad idea for George to buy the computer; he won't be able to pay his rent. But he has enough money to do so; the computer costs \$2000, and he has \$6000 in the bank. So (10) is true. Admittedly there are things close to (10) that aren't true. He hasn't got enough money to buy the computer and pay his rent. You might say that he hasn't got enough money to buy the computer given his other financial obligations. But none of this undermines (10). The point of this little story is to respond to an argument Blome-Tillmann makes towards the end of his paper. Here is how he puts the argument. (Again I've changed the numbering and some terminology for consistency with this paper.)

\begin{quote}
\noindent Suppose that John and Paul have exactly the same evidence, while John is in a low-stakes situation towards $p$ and Paul in a high-stakes situation towards $p$. Bearing in mind that IRI is the view that whether one knows $p$ depends on one's practical situation, IRI entails that one can truly assert:

\numbex{11}{
\item John and Paul have exactly the same evidence for $p$, but only John has enough evidence to know $p$, Paul doesn't.
}
\end{quote}

\noindent And this is meant to be a problem, because (12) is intuitively false.

But IRI doesn't entail any such thing. Paul does have enough evidence to know that $p$, just like George has enough money to buy the computer. Paul can't know that $p$, just like George can't buy the computer, because of their practical situations. But that doesn't mean he doesn't have enough evidence to know it. So, contra Blome-Tillmann, IRI doesn't entail this problematic conjunction.

In a footnote attached to this, Blome-Tillmann tries to reformulate the argument.

\begin{quote}
\noindent I take it that having enough evidence to `know $p$' in $C$ just means having evidence such that one is in a position to `know $p$' in $C$, rather than having evidence such that one `knows $p$'. Thus, another way to formulate (12) would be as follows: `John and Paul have exactly the same evidence for $p$, but only John is in a position to know $p$, Paul isn't.'
\end{quote}

\noindent The `reformulation' is obviously bad, since having enough evidence to know $p$ isn't the same as being in a position to know it, any more than having enough money to buy the computer puts George in a position to buy it. But might there be a different problem for IRI here? Might it be that IRI entails (13), which is false?

\numbex{12}{
\item John and Paul have exactly the same evidence for $p$, but only John is in a position to know $p$, Paul isn't.
}

\noindent There isn't a problem with (13) because almost any epistemological theory will imply that conjunctions like that are true. In particular, any epistemological theory that allows for the existence of defeaters to not supervene on the possession of evidence will imply that conjunctions like (13) are true. Any such theory, or indeed any such non-supervenience claim, will be controversial. But there are so many plausible violations of this supervenience principle, that it would be implausible if none of them work. Here are three putative examples of cases where subjects have the same evidence but different defeaters; I expect most epistemologists will find at least one of them plausible.

\begin{quote}
\noindent \textit{Logic and the Oracle} \\ Graham, Crispin and Ringo have an audience with the Delphic Oracle, and they are told $\neg p \vee q$ and $\neg \neg p$. Graham is a relevant logician, so if he inferred $p \wedge q$ from these pronouncements, his belief in the invalidity of disjunctive syllogism would be a doxastic defeater, and the inference would not constitute knowledge. Crispin is an intuitionist logician, so if he inferred $p \wedge q$ from these pronouncements his belief in the invalidity of double negation elimination would be a doxastic defeater, and the inference would not constitute knowledge. Ringo has no deep views on the nature of logic. Moreover, in the world of the story classical logic is correct. So if Ringo were to infer $p \wedge q$ from these pronouncements, his belief would constitute knowledge. Now Graham's and Crispin's false beliefs about entailment are not $p \wedge q$-relevant evidence, so all three of them have the same $p \wedge q$-relevant evidence, but only Ringo is in a position to know $p \wedge q$. \smallskip

\noindent \textit{Missing iPhone} (after \citet{Harman1973}). \\ Lou and Andy both get evidence $E$, which is strong inductive evidence for $p$. If Lou were to infer $p$ from $E$, his belief would constitute knowledge. Andy has just missed a phone call from a trusted friend. The friend left a voicemail saying $\neg p$, but Andy hasn't heard this yet. If Andy were to infer $p$ from $E$, his friend's phone call and voicemail would constitute defeaters, so he wouldn't know $p$. But phone calls and voicemails you haven't got aren't evidence you have. So Lou and Andy have the same $p$-relevant evidence, but only Lou is in a position to know $p$. 

\smallskip \noindent \textit{Fake Barns and Motorcycles} (after \citet{Gendler2005}) \\ Bob and Levon are travelling through Fake Barn Country. Bob is on a motorcycle, Levon is on foot. They are in an area where the barns are, surprisingly, real for a little ways around. On his motorcycle, Bob will soon come to fake barns, but Levon won't hit any fakes for a long time. They are both looking at the same real barn. If Bob inferred it was a real barn, not a fake, the fakes he is speeding towards would be defeaters. But Levon couldn't walk that far, so those barns don't defeat him. So Bob and Levon have the same evidence, but only Levon is in a position to know that the barn is real.
\end{quote}

\noindent I'm actually not sure what plausible theory would imply that what different agents are in a position to know depends on \textit{nothing} except for what evidence they have. The only theory that I can imagine with that consequence is the conjunction of evidentialism about justification and a justified true belief theory of knowledge. So really there's no reason to think that implying sentences like (13) is a mark against a theory.

It's been suggested to me\footnote{Footnote deleted for blind review.} that there are other more problematic conjunctions in the neighbourhood. For instance, we might worry that IRI implies that (14) is true.

\numbex{13}{
\item John and Paul are alike in every respect relevant to knowledge of $p$, but John is in a position to know $p$, and Paul isn't.
}

\noindent That would be problematic, but there's no reason to think IRI implies it. Indeed, IRI entails that the first conjunct is false, since John and Paul are unlike in one respect that IRI loudly insists is relevant. Perhaps we can do better with (15).

\numbex{14}{
\item John and Paul are alike in every respect relevant to knowledge of $p$ except their practical interests, but John is in a position to know $p$, and Paul isn't.
}

\noindent That is something IRI implies, but it seems more than a little question-begging to use its alleged counterintuitiveness against IRI. After all, it's simply a statement of IRI itself. If someone had alleged that IRI should be accepted because it was so intuitive, I guess noting how odd (15) looks would be a response to them. But that's not the way IRI is defended in the books and papers I cited in the introduction. It is defended by noting what a good job it does of handling difficult puzzles, especially puzzles concerning lotteries and conjunctions, and I don't see any reason to think there is any solution to those puzzles that isn't counterintuitive. 

On a more positive note, I think it will be worthwhile going forward to think about `afford' in these debates, since `afford' is interest-relative, at least on one disambiguation. A kind of Lewisian contextualism about `afford' would be crazy. We shouldn't say that if my rent is \$5000, and it is due tomorrow, then (9) is false, because after all, in my context someone with Ringo's money couldn't buy the computer and meet their financial obligations. So there might be some interesting patterns that `afford' has, and checking whether `knows' behaves the same way could be a good check on whether `knows' is interest-relative.
