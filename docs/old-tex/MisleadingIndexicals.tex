\footnote{Thanks to Europa Malynicz, Adam Sennet and Ted Sider for helpful comments.}

In `Now the French are invading England' Komarine Romdenh-Romluc \citeyearpar{KRR2002} offers a new theory of the relationship between recorded indexicals and their content. Romdenh-Romluc's proposes that Kaplan's basic idea, that reference is determined by applying a rule to a context, is correct, but we have to be careful about what the context is, since it is not always the context of utterance. A few well known examples illustrate this. The `here' and `now' in `I am not here now' on an answering machine do not refer to the time and place of the original utterance, but to the time the message is played back, and the place its attached telephone is located. Any occurrence of `today' in a newspaper or magazine refers not to the day the story in which it appears was written, nor to the day the newspaper or magazine was printed, but to the cover date of that publication.

Still, it is plausible that for each (token of an) indexical there is a salient context, and that `today' refers to the day of its context, `here' to the place of its context, and soon. Romdenh-Romluc takes this to be true, and then makes a proposal about what the salient context is. It is `the context that \textit{Ac} would identify on the basis of cues that she would reasonably take \textit{U} to be exploiting'. \citeyear[39]{KRR2002} \textit{Ac} is the relevant audience, `the individual who it is reasonable to take the speaker to be addressing', and who is assumed to be linguistically competent and attentive. (So \textit{Ac} might not be the person \textit{U} intends to address. This will not matter for what follows.) The proposal seems to suggest that it is impossible to trick a reasonably attentive hearer about what the referent of a particular indexical is. Since such trickery does seem possible, Romdenh-Romluc's theory needs (at least) supplementation. Here are two examples of such tricks.

\begin{quote}

{\itshape Example One} \\
Imagine that at my university, the email servers are down, so all communication from the office staff is by written notes left in our mailboxes. I notice that one of my colleagues, Bruce, has a rather full mailbox, and hence must not have been checking his messages for the last day or two. I also know that Bruce is a forgetful type, and if someone told him that he'd forgotten about a faculty meeting yesterday, he'd probably believe them. In fact he hasn't forgotten; the meeting is for later today. So I decide to play a little trick on him. I write an official looking note saying `There is a faculty meeting today', leave it undated, and put it in Bruce's mailbox underneath several other messages, so it looks like it has been there for a day or two. When Bruce sees it he is appropriately tricked, and for an instant panics about the meeting that he has missed.
\end{quote}

It seems to me that what I wrote on the note was \textit{true}. It was horribly misleading, to be sure, but still \textit{true}. And as a few people have pointed out over the years, most prominently Bill Clinton I guess, it is possible to mislead people with the truth. But on Romdemh-Romluc's proposal, what I said was false, since my audience (Bruce) reasonably took the context to be a day earlier in the week.

\begin{quote}
{\itshape
Example Two} \\
This example is closely based on a recent TV commercial. Jack leaves the following message on Jill's answering machine late one Saturday night. `Hi Jill, it's Jack. I'm at Rick's. This place is wild. There's lots of cute girls here, but I'm just thinking about you.' In the background loud music is playing, as if Jack were at a nightclub, indeed as if Jack were at Rick's, so Jill reasonably concludes that Jack was at Rick's when he sent the message, and hence that `here' refers to Rick's. In fact Jack was home alone, but wanted to hide this fact, so he turned the stereo up to full volume while leaving the message. Despite the fact that a reasonable and attentive member of the target audience inferred on the basis of contextual clues left by Jack that the context was Rick's, it was not. The context was Jack's house, and `here' in Jack's message referred to his house. Jack's trick may be less morally reprehensible than mine, but at least I managed to avoid lying, something Jack failed to do.
\end{quote}

\noindent In Example One I said something true even though what the hearer took me to say was false. In Example Two Jack says something false, though what the hearer takes him to say may well be true, assuming that there are a lot of cute girls at Rick's. Romdenh-Romluc's theory predicts that neither of these things is possible, so it does not work as it stands. This, of course, is not to say that anyone else (myself included) has a \textit{better} theory readily available, so it is unclear whether the right lesson to draw from these examples is that Romdenh-Romluc's theory needs to have some epicycles added, or that we need to try a rather different approach. One simple epicycle makes the theory extensionally adequate, but philosophically uninteresting. Consider modifying the theory to require \textit{Ac} to be not just reasonable and attentive, but informed of \textit{U}'s circumstances. Then the context identified by \textit{Ac} will be the salient context for determining the referent of \textit{U}'s indexicals. But saying this is not to offer a theory of content for recorded indexicals, it is merely to say that ideally placed observers have access to all the relevant semantic facts. Even this might be wrong if epistemicism about vagueness is correct, but if that is true then Romdenh-Romluc's theory is probably radically mistaken, for then there are facts about content that cannot be reasonably believed, even by an attentive and informed observer. We still seem to be a fair distance from having an acceptable theory.

%{\raggedleft\itshape
%Brown University
%\par}

%{\raggedleft\itshape
%Providence, RI 02912-1918 USA
%\par}

%{\raggedleft\itshape
%brian\_weatherson@brown.edu
%\par}

%
%{\itshape
%References}

%{\sffamily
%\textrm{Romdenh-Romluc, K. 2002. Now the French are invading England. }\textrm{\textit{Analysis}}\textrm{ 62: 34\nobreakdash-41.}}
